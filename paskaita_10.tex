\section{Funkcijų iškilumas}

\begin{defn}[Funkcijos grafikas]
  \begin{equation*}
    G(f) := \left\{ (x; f(x)) : x \in A \right\}.
  \end{equation*}
\end{defn}

\begin{defn}[Taškas yra žemiau]
  Taškas $(x; y)$ yra žemiau taško $(x; Y)$, jei $y \leq Y$.
\end{defn}

\begin{defn}[Iškila funkcija]
  Funkcija $f$ vadinama iškila (į apačią) intervale $(a; b)$, jei
  $\forall x_{1}, x_{2} (x_{1},x_{2} \in (a; b))$ funkcijos grafiko
  taškai $\left\{ (x; f(x)) : x_{1} \leq x \leq x_{2} \right\}$
  yra žemiau funkcijos $f$ stygos, kuri jungia funkcijos 
  grafiko taškus $(x_{1}; f(x_{1}))$ ir $(x_{2}; f(x_{2}))$.

  Fiksuojame $x$ ir $y \leq Y$, kur $y$ yra funkcijos $f$ reikšmė 
  taške $x$, o $Y$ yra stygos reikšmė taške $x$. Tada:

  \begin{equation*}
    f(x) \leq 
      \frac{f(x_{2}) - f(x_{1})}{x_{2} - x_{1}}(x - x_{1}) + f(x_{1})
  \end{equation*}
\end{defn}

\begin{exmp}
  Iškilos (į apačią) funkcijos grafiko pavyzdys: TODO: $f(x) = x^{2}$.
\end{exmp}

\begin{defn}[Iškila funkcija]
  Funkcija $f$ yra iškila intervale $(a; b)$, jei
  $\forall x, x_{1}, x_{2} (x, x_{1}, x_{2} \in (a; b)$ ir
  $x_{1} < x < x_{2})$ teisinga nelygybė:
  \begin{equation*}
    \frac{f(x) - f(x_{1})}{x - x_{1}} 
    \leq \frac{f(x_{2}) - f(x_{1})}{x_{2} - x_{1}}
  \end{equation*}
\end{defn}

\begin{defn}[Iškilos funkcijos kriterijus]
  Iškilos į apačią funkcijos kriterijus:
  \begin{equation*}
    \frac{f(x) - f(x_{1})}{x - x_{1}} 
    \leq \frac{f(x_{2} - f(x))}{x_{2} - x},
  \end{equation*}
  kur $x_{1} < x < x_{2}$.
\end{defn}

\begin{defn}[Įgaubtos funkcijos kriterijus]
  Įgaubtos (iškilos į viršų) funkcijos kriterijus:
  \begin{equation*}
    \frac{f(x) - f(x_{1})}{x - x_{1}}
    \leq \frac{f(x_{2}) - f(x)}{x_{2} - x},
  \end{equation*}
  kur $x_{1} < x < x_{2}$.
\end{defn}

\begin{exmp}
  Įgaubtos (iškilos į apačią) funkcijos grafiko pavyzdys: 
  TODO: $f(x) = -x^{2}$.
\end{exmp}

\begin{prop}
  Iškila funkcija yra tolydi.
\end{prop}

\begin{prop}
  TODO: Sutvarkyti.

  Tarkime, $f$ yra diferencijuojama intervale $(a; b)$. Tada $f$ yra
  iškyla intervale $(a; b)$ tada ir tik tada, kai $f'$ – didėja.

  \begin{proof}
    TODO.
  \end{proof}
\end{prop}

\begin{prop}
  Tarkime $f$ yra diferencijuojama du kartus.

  Jei $f'' \geq 0$, tai $f$ yra iškila.
  Jei $f'' \leq 0$, tai $f$ yra įgaubta.
\end{prop}

\begin{defn}[Vingio taškas]
  Taškas $x \in (a; b)$ vadinamas vingio tašku, jei jame keičiasi funkcijos
  iškilumas.
\end{defn}

\begin{prop}
  Jei $f$ yra diferencijuojama du kartus taške $c$, tai 
  $f''(c) = 0$ yra būtina vingio taško sąlyga.
\end{prop}

\begin{prop}
  Tarkime $f$ yra du kartus diferencijuojama. Taškas $c$ yra vingio
  taškas tada ir tik tada, kai $c$ keičiasi $f''$ ženklas.
\end{prop}

\section{Asimptotės}

\begin{defn}[Vertikali asimptotė]
  Tiesė $x = x_{0}$ yra funkcijos $f$ vertikali asimptotė, jei 
  \begin{equation*}
    \lim_{x \to x_{0}^{+}} f(x) = \infty \text{ arba }
    \lim_{x \to x_{0}^{-}} f(x) = \infty.
  \end{equation*}

  TODO: Pavyzdį su brėžiniu.
\end{defn}

\begin{defn}[Nevertikali asimptotė]
  Tiesę $y = ax + b$ vadiname nevertikalia asimptote, tada ir tik tada, kai
  \begin{equation*}
    \lim_{x \to +\infty} (f(x) - (ax + b)) = 0 \text{ arba }
    \lim_{x \to -\infty} (f(x) - (ax + b)) = 0
  \end{equation*}

  \begin{note}
    Funkcija gali turėti ne daugiau, nei dvi nevertikalias asimptotes. 
    Vieną, kai $x \to +\infty$ ir kitą, kai $x \to -\infty$.
  \end{note}

  \begin{note}
    Koeficientų $a$ ir $b$ radimo metodika, kai $x \to +\infty$:
    \begin{equation*}
      \lim_{x \to +\infty} \left( \frac{f(x)}{x} - a - 
        \left( \underbrace{\frac{b}{x}}_{\to 0} \right) \right) = 0
        \implies a = \lim_{x \to +\infty} \left( \frac{f(x)}{x} \right))
    \end{equation*}
    \begin{equation*}
      b = \lim_{x \to +\infty} (f(x) - ax)
    \end{equation*}
  \end{note}
\end{defn}

\section{Funkcijos tyrimas}

Atliekant funkcijos tyrimą atliekami tokie žingsniai:
\begin{enumerate}
  \item Nustatomi taškai, kur funkcija kerta $x$ ir $y$ ašis. Patikrinama
    ar funkcija yra periodinė, lyginė, nelyginė.
  \item Ištiriami ypatingieji taškai: $-\infty, +\infty$, trūkio taškai.
    Nustatomos trūkio taškų rūšys.
  \item Ištiriamas funkcijos monotoniškumas: surandami didėjimo, mažėjimo
    intervalai bei lokalūs ekstremumai.
  \item Surandami intervalai, kur funkcija yra iškila, įgaubta bei jos
    vingio taškai.
  \item Surandamos funkcijos asimptotės.
  \item Nubraižomas funkcijos grafikas.
\end{enumerate}
