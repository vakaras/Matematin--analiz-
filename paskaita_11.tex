\chapter{Integralas}

TODO Sutvarkyti.

\begin{note}
  Toliau darome prielaidą, kad $f: I \to \RSET$, kur $I$ – intervalas.
\end{note}

\begin{notation}
  \begin{description}
    \item[$C(I)$] visų funkcijų, kurios intervale $I$ yra tolydžios, aibė.
    \item[$C^{n}(I)$] visų funkcijų, kurios intervale $I$ yra $n$ kartų 
      tolydžiai diferencijuojamos, aibė.
    \item[$D(I)$] visų funkcijų, kurios intervale $I$ neturi antros rūšies 
      trūkių aibė. Konkretaus intervalo atveju skliaustus galima praleisti
      (pavyzdžiui, $D[a; b] := D([a; b])$.)
    \item[$S(I)$] visų funkcijų, kurios intervale $I$ yra laiptinės, aibė.
  \end{description}
\end{notation}

\begin{defn}[Tolydi funkcija]
  Funkcijos riba taške lygi funkcijos reikšmei tame taške.
\end{defn}

\begin{defn}[Tolygiai tolydi funkcija]
  Funkcija $f$ yra vadinama tolygiai tolydžia aibėje $A$, jei
  \begin{equation*}
    \forall \varepsilon (\varepsilon > 0), \exists \delta : %
    | f(x') - f(x'') < \varepsilon, %
    \forall x', x'' (x', x'' \in A, |x' - x''| < \delta)
  \end{equation*}

  (Tas pats kaip ir funkcijos ribos Koši kriterijus, tik taškas į kurį
  artėja turi būti lygus funkcijos reikšmei tame taške.)
\end{defn}

\begin{prop}
  (Kantoro teorema) Tolydi  uždarame, aprėžtame intervale funkcija yra
  tolygiai tolydi.
\end{prop}

\section{Funkcijų sekos tolydus konvergavimas}

\begin{note}
  Visų funkcijų sekos funkcijų $f_{i}$ apibrėžimo sritys (sritys, kuriose
  tiriamos funkcijos) yra vienodos ir žymimos $A$.
\end{note}

\begin{defn}[„Pataškinis“ konvergavimas]
  Sakome, kad $f_{n}$ konverguoja į $f$ aibėje $A$, jei 
  $f_{n}(x_{0}) \to f(x_{0})$, kai $n \to +\infty$.

  (Jei visos reikšmių sekos konverguoja į galutinės funkcijos reikšmes.)

  \begin{notation}
    $f_{n} \to f$
  \end{notation}
\end{defn}

\begin{defn}[Tolydus konvergavimas]
  $\sup_{x \in A} | f_{n}(x) - f(x) | \to 0$, kai $n \to +\infty$.
  \begin{notation}
    $f_{n} \rightrightarrows f$
  \end{notation}
\end{defn}

\begin{prop}
  (Koši kriterijus) $\left\{ f_{n} \right\}$ konverguoja tolygiai 
  aibėje $A$ tada ir tik tada, kai:
  \begin{equation*}
    \forall \varepsilon (\varepsilon > 0) \exists N : %
      |f_{n}(x) - f_{m}(x)| < \varepsilon, %
      \forall n, m (n, m > N), \forall x (x \in A)
  \end{equation*}
\end{prop}

\begin{prop}
  Jei $f_{n}, \forall n(n \in \NSET)$ yra tolydžios aibėje $A$, tai ir
  $f (f_n \rightrightarrows f)$ irgi yra tolydi aibėje A.
\end{prop}

\section{Kažkas naujo}

\begin{defn}[Laiptinė funkcija]
  Funkcija $f$ vadinama laiptine intervale $I$, jei intervalą $I$ galima
  suskaidyti į baigtinį skaičių intervalų ${I_{k}, k=1, 2, 3,\ldots, n}$
  (tai yra $\cup_{k=1}^{n} I_{k} = I$ ir 
  $I_{k} \cap I_{l} = \emptyset, \forall k, l (k \neq l)$), kuriuose
  funkcija turi pastovią reikšmę.
\end{defn}

\begin{prop}
  \begin{enumerate}
    \item Bet kokiai funkcijai $f (f \in D[a; b])$ egzistuoja laiptinių
      funkcijų seka 
      $\left\{ \varphi_{n} \right\} (\varphi \in S[a; b], \forall n)$,
      tolygiai konverguojanti į $f$ intervale $[a; b]$.
    \item Kiekvienos funkcijos $f (f \in D[a; b])$ trūkio taškų aibė
      yra baigtinė arba skaiti.
    \item Kiekviena funkcija $f (f \in D[a; b])$ yra aprėžta.
  \end{enumerate}
\end{prop}

\begin{prop}
  (Neapibrėžtinių integralų savybės) 17 psl.
  Pavyzdžiai 18 psl.
\end{prop}

