\chapter{Teiloro formulė}
$f(x) = f(x_0) + \frac{f'(x_0)}{1!}(x - x_0)^1 + \cdots
        + \frac{f^{(n)}(x_0)}{n!}(x - x_0)^n + r_n(x)$ (liekamasis narys)
\begin{prop}
  Tegu f n + 1 diferencijuojama intervale (a; b), x, x_0 \in (a; b)
  tada $c \in (x; x_0) (arba c \in (x_0; x)
        : r_n(x) = \frac{f^{(n + 1)}{c}a}{(n + 1)!}(x - x_0)^{n + 1}$
  \begin{proof}
    $\phi(t) = f(x) - f(t) - \frac{f'(t)}{1!}(x - t) - \cdots
               - \frac{f^{(n)}(t)}{n!}(x - t)^n$
    $\phi$ yra vieną kartą diferencijuojama pagal kintamąjį $t$,
    nes $f^{(n + 1)}(t)$ egzistuoja.
    $\phi'(t) = 0 - f'(t) - \frac{f''(t)}{1!}(x - t) 1 \frac{f'(t)}{1!}
                - \frac{f'''(t)}{2!}(x - t)^2
                - \frac{f''(t)}{2!} \cdot 2 \cdot (x - t) - \cdots
                - \frac{f^{(n + 1)}}{n!}(x - t)^n
              = - \frac{f^{(n + 1)}}{n!}(x - t)^n$
    Paimkime naują funkciją $g$, kuri vieną kartą diferencijuojama.
    $\phi$ ir $g$ tenkina Koši vidutinių reikšmių teormą
    $\implies \exists c \in (x; x_0) : \frac{\phi(x) - \phi(x_0)}{g(x) - g(x_0)} = \frac{\phi'(c)}{g'(c)} (*)$
    \[
      \phi(x_0) = r_n(x)
      \phi(x) = 0
      \phi'(t) = - \frac{f^{n + 1}(t)}{n!}(x - t)^n
    \]
    Pasirenkame $g(t) = (x - t)^{n + 1}
    g'(t) = (n + 1)(x - t)^n
    g(x) = 0
    g(x_0) = (x - x_0)^{n + 1}$
    $\phi$ ir $g$ reikšmes surašome į $(*)$
    $\frac{0 - r_n(x)}{-(x - x_0)^{n + 1}}
      = \frac{\frac{f^{(n + 1)}(c)}{n!}(x - c)^n}{-(n + 1)(x - c)^n}
      r_n(x) = \frac{f^{(n + 1)}(c)}{n!(n + 1)}(x - x_0)^{n + 1} = (n + 1)!$
  \end{proof}
\end{prop}

\begin{prop}
  Jei $f$ yra $n$ kartų diferencijuojama taško x_0 aplinkoje ir $f^{(n)}$
  yra tolydi taško $x_0$ aplinkoje, tai
  $r_n(x) = o((x - x_0)^n)$, kai $x \to x_0$
\end{prop}

Užd.
$f(x) = e^x
x_0 = 0
f^{(n)}(x) = e^x
f^{(n)}(0) = 1
f(x) = 1 + \frac{1}{1!}x^1 + \frac{1}{2!}x^2 + \frac{1}{3!}x^3 + \cdots
       + \frac{1}{n!}x^n + r_n(x)$
Teiloro formulė
$\implies \exists c \in (0; x) : r_n(x) = \frac{e^c}{(n + 1)!}x^{n + 1}$
Tarkime, $f$ nagrinėjame aplinkoje $|x| < 1$. Kokia paklaida?
$|r_n(x)| \leq \frac{e^1}{(n + 1)!} \cdot 1 = \frac{e}{(n + 1)!}
n \in +\infty r_n(x) \to 0$

\chapter{Lokalūs ekstremumai}
\begin{defn}
  Taškas $a$ vadinamas lokaliu maksimumu, jei
  $\exists U_a : f(x) \leq f(a) \forall x \in U_a$
\end{defn}
\begin{defn}
  Taškas $a$ vadinamas lokaliu minimumu, jei
  $\exists U_a : f(x) \geq f(a) \forall x \in U_a$
\end{defn}
\begin{defn}
  Taškas $a$ vadinamas funkcijos maksimumo tašku, jei
  $f(x) \leq f(a), \forall x \in A$ ($A$ – $f$ apibrėžimo sritis)
\end{defn}
\begin{defn}
  Taškas $a$ vadinamas funkcijos minimumo tašku, jei
  $f(x) \geq f(a), \forall x \in A$ ($A$ – $f$ apibrėžimo sritis)
\end{defn}
Sakykime, $f$ yra diferencijuojama. Iš Ferma teoremos \implies būtina
lokalaus ektremumo sąlyga $f'(c) = 0$.
\grafikai \implies jei taške $a$ funkcijos $f$ išvestinė keičia ženklą, tai
taškas $a$ yra lokalaus ekstremumo taškas.

\begin{exmp}
  $f(x) = 3x^2 - 6x
  f'(x) = 6x - 6$
  1) Tikrinti būtino lokalaus ekstremumo sąlygas
  $f'(x) = 0$
  2) Taškuose, kuriuose būtino lokalaus ekstremumo sąlyga išpildyta,
  tikrinti pakankamo lokalaus ekstremumo sąlygą.
  1) $6x - 6 = 0$
     $x = 1$ – jame išpildyta lokalaus ekstremumo sąlyga
  2) $f'(x) = 6(x - 1)$ kairėje ($< 1$) neigiama, dešinėje ($> 1$) teigiama
  Patikrinta: 1 yra lokalaus minimumo taškas.
\end{exmp}
\begin{exmp}
  $f(x) = |x|
  \not \exists f'(0)$
  1) $f'(x) = 
      \begin{cases}
        1 & \text{, } x > 0
        \text{neegzistuoja} & \text{, } x = 0
        -1 & \text{, } x < 0
      \end{cases}
     $
     Taške $x = 0$ sprendinių nėra, $f$ yra nediferencijuojama
     \implies taške $0$ apibrėžtų kriterijų taikyti begalima.
  dar kažkas
\end{exmp}
