\begin{prop}
  Jei $\lim_{x \to a} f(x) = b$ ir $\lim_{x \to a} f(x) = c$, tai $b = c$.
  \begin{proof}
    Tarkime priešinai $b \neq c$. Tegu $b < c$.

    Pagal \ref{limfed} apibrėžimą:
    \begin{align*}
      \lim_{x \to a} f(x) = b \iff&% 
        \forall U_{b}, \exists U_{a} :%
        f(x) \in U_{b}, \forall x(x \in U_{a}) \\
      \lim_{x \to a} f(x) = c \iff&% 
        \forall U_{c}, \exists U'_{a} :%
        f(x) \in U_{c}, \forall x(x \in U'_{a})
    \end{align*}

    Pastebėkime, kad $U_{a} \cap U'_{a}$ bus taško $a$ aplinka (taško 
    aplinka visada yra netuščia aibė).
    Fiksuojame tokias $U_{b}$ ir $U_{c}$, kurioms 
    $U_{b} \cap U_{c} = \emptyset$. Bet 
    $\forall x (x \in U_{a} \cap U'_{a}) \implies% 
      f(x) \in U_{b} \land f(x) \in U_{c}$, gavome prieštarą.
  \end{proof}
\end{prop}

\begin{prop}
  Jei $f(x) \leq g(x), \forall x (x \in U_{a})$ ir $f$ bei $g$ turi ribą
  taške $a$, tai $\lim_{x \to a} f(x) \leq \lim_{x \to a} g(x)$.
  %TODO: Įrodyti!
\end{prop}

\begin{prop}
  Jei $\lim_{x \to a} f(x) = b < c$, tai 
  $\exists U_{a} : f(x) < c, \forall x (x \in U_{a})$.
  %TODO: Įrodyti!
\end{prop}

\begin{prop}
  Jei $h(x) \leq f(x) \leq g(x), \forall x(x \in A)$ ir 
  $\lim_{x \to a} h(x) = \lim_{x \to a} g(x) = b$, tai
  $\lim_{x \to a} f(x) = b$.
  %TODO: Įrodyti!
\end{prop}

\begin{prop}
  Jei $f$ yra monotoniška intervale $I$ ir $a \in I$, tai 
  $\exists \lim_{x \to a} f(x)$. Jei papildomai žinome, kad $f$ yra
  aprėžta intervale $I$, tai $\lim_{x \to a} f(x)$ yra baigtinė.
  %TODO: Įrodyti!
\end{prop}

\begin{prop}
  Jei $f$ ir $g$ turi baigtines ribas taške $a$ ir 
  $\alpha, \beta \in \RSET$, tai 
  \begin{enumerate}
    \item \[
      \lim_{x \to a} (\alpha f(x) + \beta g(x)) =%
        \alpha \lim_{x \to a} f(x) + \beta \lim_{x \to a} g(x);
      \]
    \item \[
      \lim_{x \to a} (f(x)g(x)) =%
        \lim_{x \to a} f(x) \lim_{x \to a} g(x);
      \]
    \item Jei 
      $g(x) \neq 0, \forall x (x \in A), \lim_{x \to a} g(x) \neq 0$, tai
      \begin{equation*}
        \lim_{x \to a} \frac{f(x)}{g(x)} =
          \frac{\lim_{x \to a} f(x)}{\lim_{x \to a} g(x)}.
      \end{equation*}
  %TODO: Įrodyti!
  \end{enumerate}
\end{prop}

\begin{defn}[Funkcijos riba iš dešinės]
  Taškas $b$ vadinamas funkcijos $f$ riba taške $a$ iš dešinės, jei:
  \[
  \forall U_{b}, \exists U_{a} :%
    f(x) \in U_{b}, \forall x (x \in U_{a} \cap (a; +\infty)).
  \]
  \begin{notation} 
    \[
    \lim_{x \to a^{+}} f(x) = b
    \]
  \end{notation}
\end{defn}

\begin{defn}[Funkcijos riba iš kairės]
  Taškas $b$ vadinamas funkcijos $f$ riba taške $a$ iš kairės, jei:
  \[
  \forall U_{b}, \exists U_{a} :%
    f(x) \in U_{b}, \forall x (x \in U_{a} \cap (-\infty; a)).
  \]
  \begin{notation} 
    \[
    \lim_{x \to a^{-}} f(x) = b
    \]
  \end{notation}
\end{defn}

TODO: Įkelti per praktiką darytus pavyzdžius.
