\chapter{Funkcijos išvestinė}

\begin{note}
  $f : A \to \RSET, A \subset \RSET, a(a \in A)$ yra vidinės aibės 
  $A$ taškas.
\end{note}

\begin{defn}[Funkcijos išvestinė]
  \begin{equation*}
    f'(a) := \lim_{x \to a} \frac{f(x) - f(a)}{x - a}
  \end{equation*}
\end{defn}

\begin{notation}
  \begin{align*}
    f'(a) 
    &\equiv \frac{d f(a)}{dx} \\
    &\equiv \left. \frac{d f(x)}{dx} \right|_{x = a} \\
    &\equiv \left. f'(x) \right|_{x = a}.
  \end{align*}
\end{notation}

Jei $f'(a) \in \RSET$ sakome, kad funkcija $f$ taške $a$ yra 
diferencijuojama.

\begin{exmp}
  $f(x) = x^{2}$. Rasti $f'(x)$.

  \begin{align*}
    (x^{2})'|_{x=a} 
    &= \lim_{x \to a} \frac{x^{2} - a^{2}}{x - a} \\
    &= \lim_{x \to a} \frac{(x - a)(x + a)}{x - a} \\
    &= \lim_{x \to a} (x + a)\\
    &= 2a
  \end{align*}

  Taigi 
  \begin{equation*}
    (x^{2})'|_{x = a} = 2a
  \end{equation*}
  arba
  \begin{equation*}
    (x^{2})' = 2x.
  \end{equation*}
\end{exmp}

\begin{exmp}
  $f(x) = |x|$. Rasti $f'(0)$.

  \begin{align*}
    (|x|)'|_{x = 0} 
    &= \lim_{x \to 0} \frac{|x| - 0}{x - 0} \\
    &= \lim_{x \to 0} \frac{|x|}{x} \\
    &= \lim_{x \to 0}
    \begin{cases}
      \frac{x}{x}, & x \geq 0 \\
      -\frac{x}{x}, & x < 0
    \end{cases} \\
    &= \lim_{x \to 0}
    \begin{cases}
      1, & x \geq 0 \\
      -1, & x < 0
    \end{cases}
  \end{align*}
  Riba neegzistuoja $\implies$ išvestinė taške 0 neegzistuoja.
\end{exmp}

\begin{defn}[Diferencijuojama taške funkcija]
  Funkcija $f$ vadinama diferencijuojama taške $a$, jei 
  $\exists c (c \in \RSET)$ toks, kad 
  \begin{equation*}
    f(a + h) - f(a) = c \cdot h + o(h), h \to 0.
  \end{equation*}
  $c \cdot h$ vadinamas funkcijos $f$ diferencialo reikšme taške $a$
  su pokyčiu $h$.
\end{defn}

\begin{defn}[Diferencijuojama taške funkcija]
  FIXME: Palikti tik vieną apibrėžimą.

  Funkcija $f(x)$ vadinama diferencijuojama taške $a (a \in A)$, jei 
  jos pokytis tame taške $\Delta y (\Delta y = f(a + \Delta x) - f(a))$, 
  atitinkantis jos argumento pokytį $\Delta x$, gali būti užrašomas taip:
  \begin{equation*}
    \Delta y = c \cdot \Delta x + \alpha (\Delta x) \cdot \Delta x,
  \end{equation*}
  kur $c$ – baigtinis realusis skaičius, nepriklausantis nuo $\Delta x$;
  $\alpha(\Delta x)$ – $\Delta x$ funkcija, kuri yra be galo mažėjantis
  dydis, kai $\Delta x \to 0$ 
  ($\alpha(\Delta x) = o(\Delta x), \Delta x \to 0$).
\end{defn}

\begin{notation}
  \begin{equation*}
    d_{h} f(a) \equiv d f(a)
  \end{equation*}
\end{notation}

\begin{defn}[Funkcijos diferencialas]
  Funkcijos $f$ diferencialu atitinkančiu argumento pokytį $h$ vadiname
  funkciją $x \to d_{h} f(x)$.
\end{defn}

\begin{prop}
  Funkcija $f$ yra diferencijuojama taške $a$ tada ir tik tada, jei
  $\exists$ baigtinė $f'(a)$. Tada $d_{h} f(a) = f'(a) \cdot h$.
  \begin{proof}
    \begin{description}
      \item[Būtinumas] Tegul funkcija $f$ yra diferencijuojama taške 
        $a$. Tada teisinga:
        \begin{equation*}
          \Delta y = c \cdot \Delta x + \alpha(\Delta x) \cdot \Delta x.
        \end{equation*}
        Abi lygybės puses padalinę iš $\Delta x (\Delta x \neq 0)$ gauname:
        \begin{equation*}
          \frac{\Delta y}{\Delta x} = c + \alpha(\Delta x).
        \end{equation*}
        Kai $\Delta x$ artėja į 0, gauname
        \begin{align*}
          \lim_{\Delta x \to 0} \frac{\Delta y}{\Delta x}
          &= \lim_{\Delta x \to 0} (c + \alpha(\Delta x))
          &= c,
        \end{align*}
        o tai ir reiškia, jog taške $a$ egzistuoja funkcijos $f$ išvestinė.
      \item[Pakankamumas] Tegul funkcija $f$ taške $a$ turi baigtinę 
        išvestinę. Tada:
        \begin{equation*}
          \lim_{\Delta x \to 0} \frac{\Delta y}{\Delta x} = f'(a).
        \end{equation*}

        TODO: Užbaigti įrodymą. Idėja: pažymime $c = f'(a)$ ir prie
        dešinės pridedame $\alpha(\Delta x)$, kuri riboje nieko nekeičia,
        nes labai greitai mažėja į 0. Tada abi lygybės puses padauginame 
        iš $\Delta x$ ir gauname tai ko mums reikia.
    \end{description}
  \end{proof}
\end{prop}

\begin{notation}
  \begin{equation*}
    h \equiv \underbrace{dx}_{\mathclap{
      \text{Kiek norima mažas argumento pokytis.}}}
  \end{equation*}
\end{notation}

\begin{note}
  \begin{equation*}
    f'(x) dx = \frac{d f(x)}{d x} d x = d f(x)
  \end{equation*}
\end{note}

\section{Išvestinės savybės}

\begin{note}
  FIXME: Užrašyti normaliai.

  Jei $f$ yra diferencijuojama taške $a$, tai $f$ yra tolydi taške $a$:
  \begin{align*}
    \underbrace{f(x) - f(a)}_{x \to a} 
      &= f'(a)\underbrace{x - a}_{\to 0} + 
      \underbrace{o(x - a)}_{\to a}(x - a) \\
    f(x) - f(a) &\to 0
  \end{align*}
  Iš to seka, jog:
  \begin{equation*}
    \lim_{x \to a} f(x) = f(a),
  \end{equation*}
  tai yra, kad $f$ – tolydi.
\end{note}

\begin{prop}
  Jei $f$ ir $g$ yra diferencijuojamos taške $a$ bei 
  $\alpha \in \RSET$, tai $\alpha \cdot f$, $f + g$, $f \cdot g$ ir
  $\frac{f}{g}$ (jei $g(a) \neq 0$) yra diferencijuojamos ir teisinga:
  \begin{enumerate}
    \item 
      \begin{equation*}
        (\alpha f)' = \alpha f'
      \end{equation*}
    \item 
      \begin{equation*}
        (f + g)' = f' + g'
      \end{equation*}
    \item
      \begin{equation*}
        (f \cdot g)' = f' \cdot g + f \cdot g'
      \end{equation*}
    \item
      \begin{equation*}
        \left( \frac{f}{g} \right)' = \frac{f' \cdot g - f \cdot g'}{g^2}
      \end{equation*}
  \end{enumerate}

  \begin{proof}
    TODO: Įdėti visus įrodymus.
    
    Įrodykime, jog 
    \begin{equation*}
      (f \cdot g)'(a) = f'(a) \cdot g(a) + f(a) \cdot g'(a).
    \end{equation*}

    \begin{align*}
      (f \cdot g)'(a) 
      &= \lim_{x \to a} \frac{(f \cdot g)(x) - (f \cdot g)(a)}{x - a} 
        & \left\{ \text{Išvestinės apibrėžimas.} \right\} \\
      &= \lim_{x \to a} \frac{f(x)g(x) - f(a)g(a)}{x - a} \\
      &= \lim_{x \to a} 
        \frac{f(x)g(x) - f(x)g(a) + f(x)g(a) - f(a)g(a)}{x - a} \\
      &= \lim_{x \to a} f(x) \cdot \frac{g(x) - g(a)}{x - a} + 
        \lim_{x \to a} g(x) \cdot \frac{f(x) - f(a)}{x - a} \\
      &= \lim_{x \to a} f(x) \lim_{x \to a} \frac{g(x) - g(a)}{x - a} +
        \lim_{x \to a} g(x) \lim_{x \to a} \frac{f(x) - f(a)}{x - a} \\
      &= f(a)g'(a) + g(a)f'(a)
    \end{align*}
  \end{proof}
\end{prop}

\begin{prop}
  Jei $g$ yra diferencijuojama taške $a$ ir $f$ yra diferencijuojama 
  taške $b = g(a)$, tai
  \begin{equation*}
    (f \circ g)'(a) = f'(b) \cdot g'(a).
  \end{equation*}

  Analogiškas žymėjimas:
  \begin{equation*}
    (f \circ g)'(a) = f'(x)|_{x = g(a)} \cdot g'(x)|_{x = a}.
  \end{equation*}
\end{prop}

\begin{exmp}
  Raskime $\sin x^{2}$ išvestinę.

  Pažymėkime
  \begin{align*}
    f(x) &= \sin x \\
    g(x) &= x^{2}.
  \end{align*}
  Tada 
  \begin{equation*}
    (f \circ g)(x) = \sin x^{2}.
  \end{equation*}
  Dabar galime pritaikyti išvestinės skaičiavimo formulę:
  \begin{align*}
    g'(x) &= (x^{2})' = 2x \\
    f'(x) &= (\sin x)' = \cos x \\
    (f \circ g)'(a) &= (\cos x)|_{x = a^{2}} \cdot (2x)|_{x=a} 
      = (\cos a^2)(2a).
  \end{align*}

  Taigi
  \begin{equation*}
    (\sin x^{2})' = 2x \cos x^{2}
  \end{equation*}

\end{exmp}

\section{Elementariųjų funkcijų išvestinės}

Sąrašą žr. \ref{dx_formulynas}.

\begin{exmp}
  \begin{align*}
    (\sin x)'|_{x=a}
    &= \lim_{x \to a} \frac{\sin x - \sin a}{x - a} \\
    &= \lim_{x \to a} \frac{
      2 \sin \left( \frac{x - a}{2} \right) 
      \cos \left( \frac{x + a}{2} \right)
      }{x - a} \\
    &= \lim_{x \to a} 
      \underbrace{
        \frac{\sin \left( \frac{x - a}{2} \right)}{\frac{x - a}{2}}
      }_{\to 1}
      \frac{2 \cos \left( \frac{x + a}{2} \right)}{2} \\
      &= \lim_{x \to a} \cos \frac{x + a}{2} \\
      &= \cos a
  \end{align*}

  Taigi
  \begin{equation*}
    (\sin x)' = \cos x, \forall x (x \in \RSET).
  \end{equation*}
\end{exmp}

\begin{exmp}
  \begin{align*}
    (\ln x)'|_{x = a}
    &= \lim_{x \to a} \frac{\ln x - \ln a}{x - a} \\
    &= \lim_{x \to a} \frac{\ln \frac{x}{a}}{x - a} \\
    &= \lim_{x \to a} 
      \frac{\ln \left( 1 + \frac{x}{a} - 1 \right)}{x - a} \\
    &= \lim_{x \to a} 
      \frac{\ln \left( 1 + \frac{x - a}{a} \right)}{\frac{x - a}{a} a} \\
    &= \lim_{x \to a} 
      \frac{1}{a}
      \underbrace{
        \frac{\ln \left( 1 + \frac{x - a}{a} \right)}{\frac{x - a}{a}}
        }_{\to 1} \\
    &= \frac{1}{a}
  \end{align*}

  Taigi
  \begin{equation*}
    (\ln x)' = \frac{1}{x}, \forall x (x \in (0; +\infty)).
  \end{equation*}
\end{exmp}

\begin{exmp}
  \begin{align*}
    (e^{x})'|_{x = a} 
    &= \lim_{x \to a} \frac{e^{x} - e^{a}}{x - a} \\
    &= \lim_{x \to a} \frac{e^{a} (e^{x-a} - 1)}{x - a} \\
    &= \lim_{x \to a} e^{a}
      \underbrace{\frac{e^{x - a} - 1}{x - a}}_{\to 1} \\
    &= e^{a}
  \end{align*}

  Taigi
  \begin{equation*}
    (e^{x})' = e^{x}, \forall x (x \in \RSET)
  \end{equation*}
\end{exmp}

\begin{prop}
  (Atvirkštinės funkcijos išvestinė) Tarkime $f$ tolydi taško $a$ 
  aplinkoje, $f$ yra diferencijuojama taške $a$ ir $f'(a) \neq 0$ 
  bei taško $a$ aplinkoje egzistuoja funkcijos
  $f$ atvirkštinė funkcija ($f^{-1}$). Tada
  \begin{equation*}
    (f^{-1})'(b) = \frac{1}{f'(a)}, \text{ kur } b = f(a).
  \end{equation*}
\end{prop}

\begin{exmp}
  \begin{equation*}
    f(x) = \sin x, \frac{-\pi}{2} \leq x \leq \frac{\pi}{2}
  \end{equation*}

  \begin{align*}
    (\arcsin y)' 
    &\equiv (f ^{-1})'(y) \\
    &= \frac{1}{(\sin x)'} \\
    &= \frac{1}{\cos x} \\
    &= \frac{1}{\sqrt{1 - \sin^{2} x}} \\
    &= \frac{1}{\sqrt{1 - y^{2}}}
  \end{align*}

  Išvada:
  \begin{equation*}
    (\arcsin x)' = \frac{1}{\sqrt{1 - x^{2}}}, \forall x (x \in (-1; 1)).
  \end{equation*}
\end{exmp}

\section{Vidurinių reikšmių teoremos}

\begin{prop}
  (Ferma teorema) Tarkime $f$ yra diferencijuojama atvirame intervale
  $(a; b)$ ir tame intervale egzistuoja didžiausia arba mažiausia
  funkcijos reikšmė taške $c (c \in (a; b))$. Tada $f'(c) = 0$.

  \begin{proof}
    Tarkime $f$ taške $c$ įgyja didžiausią reikšmę. Tai yra
    \begin{equation*}
      f(c) \geq f(x), \forall x (x \in A).
    \end{equation*}

    Jei $x > c$, tada 
    \begin{align}
      f(x) - f(c) &\leq 0
      \label{_ferma_01} \\
      x - c &\geq 0
      \label{_ferma_02}.
    \end{align}

    Iš \ref{_ferma_01} ir \ref{_ferma_02} nelygybės:
    \begin{equation*}
      \lim_{x \to c^{+}} \frac{f(x) - f(c)}{x - c} \leq 0,
    \end{equation*}
    o iš to išplaukia, kad
    \begin{equation}
      f'(c^{+}) \leq 0.
      \label{_ferma_03}
    \end{equation}

    Jei $x < c$, tada
    \begin{align}
      f(x) - f(c) &\leq 0
      \label{_ferma_04} \\
      x - c &\leq 0
      \label{_ferma_05}.
    \end{align}

    Iš \ref{_ferma_04} ir \ref{_ferma_05} nelygybės:
    \begin{equation*}
      \lim_{x \to c^{-}} \frac{f(x) - f(c)}{x - c} \geq 0,
    \end{equation*}
    o iš to išplaukia, kad
    \begin{equation}
      f'(c^{-}) \geq 0.
      \label{_ferma_06}
    \end{equation}

    Iš \ref{_ferma_03} ir \ref{_ferma_06} gauname, kad $f'(c) = 0$.
  \end{proof}
\end{prop}

\begin{prop}
  (Rolio teorema) Tarkime $f$ yra tolydi uždarame intervale $[a; b]$
  bei $f$ yra diferencijuojama $(a; b)$, $f(a) = f(b)$ ir 
  $\lim_{x \to a^{+}} f(x) = f(a)$. Tada 
  $\exists c (c \in (a; b))$, kur $f'(c) = 0$.

  \begin{proof}
    Jei $f(x) \equiv const, \forall x (x \in [a; b])$, tada 
    $f'(x) = 0, \forall x (x \in (a; b))$. Todėl pakanka įrodyti 
    teoremą tik tuo atveju, kai $f \not\equiv const$.

    Kadangi $f$ tolydi ir $f(a) = f(b)$, tai $f$ intervale
    $[a; b]$ įgis didžiausią arba mažiausią reikšmę. Kadangi
    $f \not\equiv const$, tai 
    $\exists c (c \in (a; b)) : f(c) \neq f(a) \neq f(b)$, todėl
    didžiausią arba mažiausią reikšmę $f$ įgis intervale $(a; b)$.
    Todėl galime pritaikyti Ferma teoremą.
  \end{proof}
\end{prop}

\begin{prop}
  \label{lagran_vid}
  (Lagranžo teorema) Tarkime $f$ yra tolydi $[a; b]$ ir $f$ yra 
  diferencijuojama $(a; b)$. Tada 
  $\exists c (c \in (a; b)) : f(b) - f(a) = f'(c)(b - a)$.
\end{prop}

\begin{prop}
  \label{kosi_vid}
  (Koši teorema) Tarkime $f$ ir $g$ yra tolydžios intervale $[a;b]$ ir
  $f, g$ yra diferencijuojamos $(a; b)$ ir $g'(x) \neq 0$ visiems 
  $x (x \in (a; b))$. Tada $\exists c (c \in (a; b))$ toks, kad
  \begin{equation*}
    \frac{f(b) - f(a)}{g(b) - g(a)} = \frac{f'(c)}{g'(c)}.
  \end{equation*}
\end{prop}
