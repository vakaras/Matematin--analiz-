\section{Lokalūs ekstremumai}

\begin{defn}[Funkcijos lokalus maksimumas]
  Taškas $a$ vadinamas lokaliu maksimumu, jei
  \begin{equation*}
    \exists U_{a} : f(x) \leq f(a), \forall x (x \in U_{a} \cap A).
  \end{equation*}
\end{defn}

\begin{defn}[Funkcijos lokalus minimumas]
  Taškas $a$ vadinamas lokaliu minimumu, jei
  \begin{equation*}
    \exists U_{a} : f(x) \geq f(a), \forall x (x \in U_{a} \cap A)
  \end{equation*}
\end{defn}

\begin{defn}[Funkcijos maksimumas]
  Taškas $a$ vadinamas funkcijos maksimumo tašku, jei 
  $f(x) \leq f(a), \forall x (x \in A)$.
\end{defn}

\begin{defn}[Funkcijos minimumas]
  Taškas $a$ vadinamas funkcijos minimumo tašku, jei
  $f(x) \geq f(a), \forall x (x \in A)$.
\end{defn}

\begin{prop}
  Jei funkcija $f$ yra diferencijuojama, tai taškas $a$ yra jos
  lokalus ekstremumas tada ir tik tada, kai funkcijos $f$ išvestinė
  tame taške keičia ženklą.

  FiXME: Performuluoti.
  \begin{proof}
    TODO: Įrodyti. (Būtinumą galima iš Ferma teoremos.)
  \end{proof}
\end{prop}

Lokalių ekstremumų paieškos schema ($f : A \to \RSET$):
\begin{enumerate}
  \item Skaidome $A$ į dvi taškų aibes: $A'$ ir $A''$, kur
    $A' \cup A'' = A$ ir $A' \cap \A'' = \emptyset$.
    $A'$ – taškai, kur $f$ yra diferencijuojama, o 
    $A''$ – taškai, kur $f$ nėra diferencijuojama.
  \item $A''$ aibės taškus analizuojame pagal apibrėžimą, o 
    $A'$ aibės taškus naudodami išvestinę.
\end{enumerate}
