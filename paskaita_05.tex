\section{Viršutinė ir apatinė ribos}

\begin{defn}[Viršutinė riba]
  \begin{equation*}
    \limsup _{x \to a} f(x) := 
      \lim _{\delta \to 0} 
      \sup \left\{ f(x), x \in A : |x - a| < \delta \right\}
  \end{equation*}
\end{defn}

\begin{defn}[Apatinė riba]
  \begin{equation*}
    \liminf _{x \to a} f(x) :=
      \lim _{\delta \to 0}
      \inf \left\{ f(x), x \in A : |x - a| < \delta \right\}
  \end{equation*}
\end{defn}

\begin{exmp}
  \begin{align*}
    \limsup _{x \to 0} \left( \sin \frac{1}{x} \right) 
    &= \lim _{\delta \to 0} \sup 
      \left\{ \sin \frac{1}{x} : |x - a| < \delta \right\} \\
    &= \lim _{\delta \to 0} 1 \\
    &= 1
  \end{align*}
  \begin{align*}
    \liminf _{x \to 0} \left( \sin \frac{1}{x} \right) 
    &= \lim _{\delta \to 0} (-1) = -1
  \end{align*}
\end{exmp}

\begin{prop}
  \begin{equation*}
    \exists \lim _{x \to a} f(x) \iff
    \limsup _{x \to a} f(x) = \liminf _{x \to a} f(x)
  \end{equation*}
\end{prop}

\begin{defn}[Viršutinė riba]
  \begin{equation*}
    \limsup _{x \to a} f(x) :=
      \lim _{\delta \to 0}
      \sup \left\{ f(x) : x 
        (x \in U(a, \delta) \cap A \setminus \{a\}) 
        \right\}
  \end{equation*}
\end{defn}

\begin{defn}[Apatinė riba]
  \begin{equation*}
    \liminf _{x \to a} f(x) :=
      \lim _{\delta \to 0}
      \inf \left\{ f(x) : x 
        (x \in U(a, \delta) \cap A \setminus \{a\}) 
        \right\}
  \end{equation*}
\end{defn}

\begin{notation}
  \begin{align*}
    \lim _{x \to a^{+}} f(x) &\equiv f(a^{+}) \\
    \lim _{x \to a^{-}} f(x) &\equiv f(a^{-}) \\
  \end{align*}
\end{notation}

\begin{exmp}
  Suskaičiuokite ribą:
  \begin{equation*}
    \limsup _{x \to 0} \frac{\sqrt{x^2} + 2x}{\sin x}
  \end{equation*}

  FIXME: Sutvarkyti modulio ženklus.

  Pažymėkime ir pertvarkykime:
  \begin{align*}
    f(x) 
    &= \frac{\sqrt{x^2} + 2x}{\sin x} \\
    &= \frac{|x| + 2x}{\sin x}, & \text{ kur } x \neq 0; \\
    f(x) 
    &=
    \begin{cases}
      \frac{3x}{\sin x}, & x \geq 0 \\
      \frac{x}{\sin x}, & x < 0
    \end{cases}.
  \end{align*}

  Iš $\lim _{x \to 0} \frac{\sin x}{x} = 1$ įrodymo 
  (\ref{lim_sinx_x} pavyzdys) žinome, kad $\sin |x| \leq |x|$ ir 
  kad $\cos |x| \leq \frac{\sin |x|}{|x|}$. Atlikę
  kelis pertvarkymus (verta pastebėti, kad visus veiksmus atliekame
  0 aplinkoje)
  \begin{align*}
    \sin |x| &\leq |x| \\
    \sqrt{1 - \cos ^{2} |x|} &\leq |x| \\
    1 - \cos^{2} |x| &\leq |x|^{2} \\
    1 - |x|^{2} &\leq \cos^{2} |x| \\
    (1 - |x|)(1 + |x|) &\leq \cos^{2} |x| \\
    (1 - |x|) &\leq \frac{\cos^{2} |x|}{1 + |x|} \\
      &\leq \cos^{2} |x| \\
      &\leq \cos |x|.
  \end{align*}
  gavome, jog:
  \begin{equation*}
    1 - |x| \leq \cos |x|.
  \end{equation*}

  Dabar galime įvertinti $|\sin x|$:
  \begin{align*}
    \cos |x| &\leq \frac{\sin |x|}{|x|}  \\
    1 - |x| &\leq \frac{\sin |x|}{|x|} \\
    |x|(1 - |x|) &\leq |\sin x| \\
    |x|(1 - |x|) &\leq |\sin x| \leq |x|.
  \end{align*}

  Taigi $\left| \frac{3x}{\sin x} \right|$ galime įvertinti:
  \begin{equation*}
    3 = \frac{3x}{x} \leq \frac{3x}{\sin x} \leq \frac{3x}{x(1 - |x|)}
    = \frac{3}{1 - x} = 3 \frac{1}{1 - x},
  \end{equation*}
  o $\left| \frac{x}{\sin x} \right|$:
  \begin{equation*}
    1 = \frac{|x|}{|x|} \leq \left| \frac{x}{\sin x} \right| \leq
    \frac{|x|}{|x|(1 - |x|)} = \frac{1}{1 - |x|}.
  \end{equation*}

  Dabar galime įvertinti
  \begin{equation*}
    h(\delta) = \sup \left\{ f(x) : x 
      (x \in U(a, \delta) \cap A \setminus \left\{ a \right\}) \right\}.
  \end{equation*}
  Gauname: (FIXME: neaiški vieta.)
  \begin{equation*}
    3 \leq h(\delta) \leq 3 \frac{1}{1 - \delta}.
  \end{equation*}

  Taigi
  \begin{equation*}
    \lim _{\delta \to 0} h(\delta) = 3.
  \end{equation*}
\end{exmp}

\section{Monotoniškos funkcijos}

\begin{prop}
  \label{lim_mon}
  $f: A \to \RSET$, $a$ – ribinis ir didžiausias $A$ taškas ir $f$ yra
  monotoniška. Tada $f$ turi ribą taške $a$. Ir jei $f$ yra ir apibrėžta,
  tai riba taške $a$ yra baigtinė.
\end{prop}

\begin{exmp}
  Įrodyti, kad $f(x), f(x) = x^2$, taške $a, a = 1$, turi baigtinę ribą.
  Apibrėžimo sritis $A = \left( \frac{1}{2}; 1 \right)$.

  \begin{proof}
    Pirma parodysime, kad $f(x)$ intervale $A$ didėja. Paimkime 
    $x_{1}$ ir $x_{2}$, tokius, kad $x_{1} < x_{2}$. Tada:

    \begin{align*}
      f(x_{2}) - f(x_{1}) 
      &= x _{2} ^{2} - x _{1} ^{2} \\
      &= \underbrace{(x_{2} - x_{1})}_{> 0}
        \underbrace{(x_{2} + x_{1})}_{> 0}
    \end{align*}

    Gavome, kad $f(x_{2}) > f(x_{1})$, o tai reiškia, kad $f(x)$ intervale
    $\left( \frac{1}{2} ; 1 \right)$ didėja. Taigi pagal \ref{lim_mon}
    teiginį $\exists \lim _{x \to a} f(x)$.

    Ar $f(x)$ yra aprėžta? Tai yra ar
    \begin{equation*}
      \exists c (c \in \RSET) : x^{2} \leq c, \forall x (x \in A)?
    \end{equation*}
    Taip $c = 1$. Taigi $\lim _{x \to 1} x^{2}$ yra baigtinė.
  \end{proof}
\end{exmp}

\section{Funkcijos tolydumas}

\begin{defn}[Tolydi taške funkcija]
  Funkcija $f$ vadiname tolydžia taške $a$, jei 
  \begin{equation*}
    \forall U_{f(a)}, \exists U_{a} : 
      f(x) \in U_{f(a)}, \forall x (x \in U_{a} \cap A)
  \end{equation*}
\end{defn}

$f(x)$ gali būti tolydi taške $a$ dviem atvejais:
\begin{enumerate}
  \item jei $a$ – izoliuotas aibės $A$ taškas, ir 
  \item jei $a$ – ribinis aibės $A$ taškas bei 
    $\lim _{x \to a} f(x) = f(a)$.
\end{enumerate}

\begin{defn}[Tolydi aibėje funkcija]
  Funkcija $f$ vadinama tolydžia aibėje $B$, jei ji yra tolydi visuose
  aibės $B$ taškuose.
\end{defn}

\begin{note}
  Toliau $f: A \to B$ ir $a \in A$.
\end{note}

\begin{defn}[Trūkio taškų tipai]
  \hfill \\
  \begin{description}
    \item[1 rūšies trūkis] Jei 
      \begin{equation*}
        \exists \lim _{x \to a^{-}} f(x) 
        \text{ ir }
        \exists \lim _{x \to a^{+}} f(x) 
      \end{equation*}
       ir jos abi yra baigtinės, tai
      taškas a yra vadinamas 1 rūšies trūkiu.
    \item[pataisomas trūkis] Jei
      \begin{equation*}
        \lim _{x \to a^{-}} f(x) = \lim _{x \to a^{+}} f(x) 
        \implies \exists \lim _{x \to a} f(x)
      \end{equation*}
      yra baigtinė, tai taškas $a$ vadinamas pataisomu trūkio tašku.
    \item[2 rūšies trūkis] Jei
      \begin{equation*}
        \nexists \lim _{x \to a^{-}} f(x)
        \text{ arba }
        \nexists \lim _{x \to a^{+}} f(x),
      \end{equation*}
      arba bent viena iš jų yra begalinė, tai taškas $a$ yra vadinamas
      2 rūšies trūkiu.
  \end{description}
\end{defn}

\begin{exmp}
  \begin{equation*}
    f(x) = \frac{x}{x}, a = 0
  \end{equation*}

  Kadangi $\lim _{x \to 0} f(x) \neq f(0)$, nes $\nexists f(0)$, tai
  taškas $x = 0$ yra trūkio.

  Kadangi $\lim _{x \to 0} \frac{x}{x} = 1$, tai $x = 0$ yra 1 rūšies
  trūkio taškas ir jis yra pataisomas.
\end{exmp}

\begin{exmp}
  \begin{equation*}
    f(x) = \sign(x), a = 0
  \end{equation*}

  Kadangi
  \begin{align*}
    \lim _{x \to 0^{+}} \sign(x) &= 1 \\
    \lim _{x \to 0^{-}} \sign(x) &= -1,
  \end{align*}
  tai $x = 0$ yra 1 rūšies trūkio taškas.
\end{exmp}

\begin{exmp}
  \begin{equation*}
    f(x) = \frac{1}{|x|}, a = 0
  \end{equation*}

  Kadangi
  \begin{align*}
    \lim _{x \to 0^{+}} \frac{1}{|x|} &= +\infty \\
    \lim _{x \to 0^{-}} \frac{1}{|x|} &= +\infty,
  \end{align*}
  tai $x = 0$ yra 2 rūšies trūkio taškas.
\end{exmp}

\begin{prop}
  Jei $f$ ir $g$ yra tolydžios taške $a$, tai ir 
  $f + g$, $f - g$, $f \cdot g$ bei $\frac{f}{g}$ (jei $g(a) \neq 0$) 
  yra tolydžios taške $a$.

  \begin{proof}
    Žymėjimas:
    \begin{equation*}
      (f + g)(x) := f(x) + g(x)
    \end{equation*}

    Įrodysime, kad jei $f$ ir $g$ yra tolydžios taške $a$, tai ir 
    $(f + g)(x)$ yra tolydi taške $a$.

    Tarkime, jog $f$ ir $g$ yra tolydžios taške $a$. Reikia įrodyti:
    \begin{equation*}
      \lim _{x \to a} (f + g)(x) = (f + g)(a)
    \end{equation*}

    Tikriname:
    \begin{align*}
      \lim _{x \to a} (f + g)(x)
      &= \lim _{x \to a} (f(x) + g(x)) \\
      &= \lim _{x \to a} f(x) + \lim _{x \to a} g(x) \\
      &= f(a) + g(a) \\
      &= (f + g)(a)
    \end{align*}
  \end{proof}
\end{prop}

\begin{prop}
  $ (f \circ g)(x) := f(g(x))$ Jei $g$ yra tolydi taške $a$ ir 
  $f$ yra tolydi taške $g(a)$, tai $f \circ g$ yra tolydi taške $a$.
\end{prop}
