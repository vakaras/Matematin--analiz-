\appendix
\chapter{Pagalbinės formulės uždaviniams spręsti}

\section{Ribų skaičiavimo formulės}

\begin{align} 
%
  &\lim _{x \to 0} \frac{\sin x}{x} = 1
  \label{f_lim_sin} \\
%
  &\lim _{x \to 0} \left( 1 + x \alpha \right)^{\frac{1}{x}} = e^{\alpha}
  \label{f_lim_exp} \\
%
  &\lim _{x \to 0} \frac{a^x - 1}{x} = \ln a, 
  & \text{ kur } a > 0, a \not{=}1
  \label{f_lim_ep} \\
% 
  &\lim _{x \to 0} \frac{(1 + x)^r -1}{x} = r, & \text{ kur } r \in \RSET
  \label{f_lim_lp} \\
%
  &\lim _{x \to 0} \frac{\log _{a} (1 + x)}{x} = \log _{a} e, 
  & \text{ kur } a > 0, a \not{=} 1
  \label{f_lim_log} \\
%
  &\lim _{x \to +\infty} \frac{a^x}{x^r} = +\infty, 
  & \text{ kur } a > 1, r \in \RSET
  \label{f_lim_rl} \\
% 
  &\lim _{x \to +\infty} \frac{\log _{a} x}{x^r} = 0,
  & \text{ kur } a > 1, r > 0
  \label{f_lim_llb} \\
%
  &\lim _{x \to 0} x^p \log _{a} x = 0, & \text{ kur } a > 1, p > 0
  \label{f_lim_lln}
%
\end{align}

