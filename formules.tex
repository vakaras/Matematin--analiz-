\appendix
\chapter{Pagalbinės formulės uždaviniams spręsti}

\section{Ribų skaičiavimo formulės}

\begin{align} 
%
  &\lim _{x \to 0} \frac{\sin x}{x} = 1
  \label{f_lim_sin} \\
%
  &\lim _{x \to 0} \left( 1 + x \alpha \right)^{\frac{1}{x}} = e^{\alpha}
  \label{f_lim_exp} \\
%
  &\lim _{x \to 0} \frac{a^x - 1}{x} = \ln a, 
  & \text{ kur } a > 0, a \neq 1
  \label{f_lim_ep} \\
% 
  &\lim _{x \to 0} \frac{(1 + x)^r -1}{x} = r, & \text{ kur } r \in \RSET
  \label{f_lim_lp} \\
%
  &\lim _{x \to 0} \frac{\log _{a} (1 + x)}{x} = \log _{a} e, 
  & \text{ kur } a > 0, a \neq 1
  \label{f_lim_log} \\
%
  &\lim _{x \to +\infty} \frac{a^x}{x^r} = +\infty, 
  & \text{ kur } a > 1, r \in \RSET
  \label{f_lim_rl} \\
% 
  &\lim _{x \to +\infty} \frac{\log _{a} x}{x^r} = 0,
  & \text{ kur } a > 1, r > 0
  \label{f_lim_llb} \\
%
  &\lim _{x \to 0} x^p \log _{a} x = 0, & \text{ kur } a > 1, p > 0
  \label{f_lim_lln}
%
\end{align}

\section{Trigonometrinės tapatybės}

\begin{align}
%
  & \sin ^{2} \alpha + \cos ^{2} \alpha = 1
  \label{f_tri_kvsum} \\
%
  & \sin \alpha - \sin \beta = 2 
    \sin \left( \frac{\alpha - \beta}{2} \right)
    \cos \left( \frac{\alpha + \beta}{2} \right)
  \label{f_tri_sinsk} \\
%
  & \cos 2 \alpha = \cos ^{2} \alpha - \sin ^{2} \alpha
  \label{f_tri_dkcos} \\
%
  & \sin 2 \alpha = 2 \sin \alpha \cos \alpha
  \label{f_tri_dksin} 
%
\end{align}
