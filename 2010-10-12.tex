\chapter{Funkcijos tolydumas}
\begin{prop}[I Vejerštraso]
  Tolydi funkcija f uždarame aprėžtame intervale yra aprėžta.
  $f : [a; b] \to \SETR a, b \neq \infty$
  \begin{proof}
    Tarkime, $f$ nėra aprėžta iš viršaus.
    Iš priešingo apibrėžimo $\implies \forall n \in \SETN \exists x_n \in [a; b] : f(x_n) > n$
    Sukonstravome $\{x_n\}, a \leq x_n \leq b$
    Iš Vejerštraso teoremos skaičių sekoms seka
    $\exists \{x_{n_k}\} x_{n_k} \to x', n_k \to +\infty$
    Iš sekų ribų elemento savybės $a \leq x_{n_k} \leq b ir x_{n_k} \to x'
    \implies a \leq x' \leq b$
    $f$ yra tolydi $\implies \lim_{x \to x'}{f(x)} = f(x')$, t. y.
    $\lim_{x_{n_k} \to x'}{f(x_{n_k})} = f(x') \neq +\infty$
    Iš $\forall n \in \SETN \exists x_n : f(x_n) > n
    \implies \lim_{x_{n_k} \to x'}{f(x_{n_k})} = +\infty$
    Gavome prieštarą.
    Vadinasi, $f$ yra aprėžta iš viršaus.
    Tą pačią įrodymo schemą pritaikome situacijai, kai $f$ nėra aprėžta iš apačios.
  \end{proof}
\end{prop}
\begin{exmp}
  Aibė $B$ vadinama aprėžta, jei $\exists c > 0 : |x| \leq c, \forall x \in \SETR$
  $\SETN$ – neaprėžta
  $(0; +\infty)$ – neaprėžta
  $[0; 1]$ – aprėžta
  $(0; 1000)$ – aprėžta

  $y = x^2  [0; 100]            \implies y = x^2$ intervale $[0; 100]$ yra aprėžta
  ↑ tolydi ↑ uždaras, aprėžta
\end{exmp}
\begin{priminimas}
  Funkcija f vadinama aprėžta aibėje $B$, jei
  $\exists c > 0 : |f(x)| \leq c, \forall x \in B$
\end{priminimas}
\begin{exmp}
  $y = \frac{1}{x} (0; 1)$
           ↑ tolydi ↑ aprėžtas, neuždaras!
\end{exmp}
\begin{exmp}
  $y = x^2 [0; +\infty)$
     ↑ tolydi ↑ uždaras, neaprėžtas!
\end{exmp}


\begin{prop}[II Vejerštraso]
  Tolydi funkcija aprėžtame ir uždarame intervale turi šiame intervale
  didžiausią ir mažiausią reikšmes.
  \begin{exmp}
    $y = \frac{1}{x} (0; 1)$
                       ↑ neuždaras
  \end{exmp}
  \begin{proof}
    Be.
  \end{proof}
\end{prop}

\begin{prop}[I Bolcmano-Koši teorema]
  Jei tolydi funkcija uždaro ir aprėžto intervalo kraštuose įgyja skirtingo
  ženklo reikšmes, tai funkcija intervalo vidiniame taške įgyja reikšmę 0.
  \begin{proof}
    \[
    f : [a; b]_{a, b \neq \infty} \to \SETR f(a) < 0, f(b) > 0
    c = \frac{b + a}{2}
    ---[---|---]-- 
       a   c   b
    \]
    $[a; c]$ arba $[c; b]$ funkcijos intervalo kraštuose turės skirtingo
    ženklo reikšmes.
    $f(c) > 0$, tai $[a; c]$
    $f(c) < 0$, tai $[c; b]$
    $f(c) = 0$ – $c$ yra ieškomas taškas
    Gautą intervalą pažymėkime $[a_1; b_1] (f(a_1) < 0, f(b_1) > 0)$
    Renkamės vieną iš $[a_1; c_1]$ ir $[c_1; b_1]$, kuris galuose turės
    skirtingų ženklų reikšmes.
    Ir t. t.
    Suformuojame įdėtųjų intervalų seką
    $[a; b] \supset [a_1; b_1] \supset [a_2; b_2] \supset ...
    b_n - a_n = \frac{b - a}{2^n} \to 0
    \implies[susitraukiančiųjų intervalų lema] \exists c' \in [a; b] :
    a_n \leq c' \leq b_n, \forall n
    a_n \to c', b_n \to c'$
    Turime $f(a_n) \leq 0, f(b_n) \geq 0$
    $f$ tolydi, todėl $\lim_{a_n \to c'}{f(a_n)} = f(c')$
    ir $\lim_{b_n \to c'}{f(b_n) = f(c')}$
    \[
      \implies f(a_n) \leq 0, \forall n
      \lim_{n \to +\infty}{f(a_n)} \equiv \lim_{a_n \to c'}{f(a_n)} \leq 0
      \implies f(c') \leq 0
    \]
  \end{proof}
\end{prop}

\begin{prop}[II Bolcmano-Koši]
  Jei uždaro aprėžto intervalo kraštuose tolydi funkcija įgyja reikšmes $c$, $d$,
  tai šiame intervale funkcija įgyja visas reikšmes iš intervalo $[c; d]$
\end{prop}

\chapter{Teoremos apie intervalo atvaizdavimą}
\begin{prop}
  Jei $f : I \to \SETR$ tolydi, $I$ – bet koks intervalas, tai $f(I)$ – intervalas
\end{prop}
\begin{prop}
  Jei $f : I \to \SETR$ tolydi, $I$ – uždaras, aprėžtas intervalas,
  tai $f(I)$ – uždaras, aprėžtas intervalas
\end{prop}
\begin{defn}
  kompaktiškas intervalas = uždaras aprėžtas intervalas
\end{defn}

\chapter{Asimptominis funkcijų įvertinimas}
\begin{defn}
  Rašysime $f(x) = o(g(x))$, jei $\lim_{x \to a}{\frac{f(x)}{g(x)}} = 0$
  (alternatyvus žymėjimas: $f(x) << g(x), x \to a$)
  \begin{exmp}
    $x^2 = o(x), x \to 0$, nes $\frac{x^2}{x} \to 0, x \to 0$
  \end{exmp}
  $f \in o(g(x))$, kai x \to a
           ↑ funkcijų aibė
  N. d.
  $o(f(x)) + o(f(x)) = o(f(x))$
  $x \to a$
  \SETZ + \SETZ = \SETZ
\end{defn}
\begin{defn}
  Rašysime $f(x) \sim g(x), x \to a$,
  jei $\lim_{x \to a}{\frac{f(x)}{g(x)}} = 1$
  \begin{exmp}
    $\sin x \sim x, x \to 0, \lim_{x \to 0}{\frac{\sin x}{x}} = 1$
  \end{exmp}
\end{defn}
\begin{defn}
  Rašysime $f(x) = O(g(x)), x \in U$,
  jei $\exists c > 0 : |f(x)| \leq c \cdost |g(x)| \forall x \in U$
\end{defn}
\begin{defn} FIXME
  Rašysime $f(x) = O(g(x)), x \to a, x \in U$,
  jei $\exists c > 0 : |f(x)| \leq c \cdost |g(x)| \forall x \in U$
\end{defn}
\begin{exmp}
...
