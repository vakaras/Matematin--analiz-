\chapter{Funkcijos riba}

\begin{notation}
  Funkcija iš realiųjų skaičių aibės $A$ į visų realiųjų skaičių aibę 
  $\RSET$ žymima:
  \[
  \begin{array}[]{c l}
    f : A \to \RSET, & \text{kur $A$ – funkcijos $f$ apibrėžimo sritis.}
  \end{array}
  \]
\end{notation}

\begin{defn}[$\varepsilon$-aplinka]
  \[
  B(\varepsilon; a) = U(\varepsilon; a) =%
  \{ x \in \RSET : |x - a| < \varepsilon \}
  \]
\end{defn}

\begin{defn}[Taško aplinka]
  Taško $a$ aplinka vadinsime aibę $A$, kuriai:
  \[
  \varepsilon > 0 : \exists U(\varepsilon; a) \subset A
  \]
\end{defn}

\begin{defn}[Atvira aibė]
  Aibė $A$ vadinama atvira aibe, jei ji yra kiekvieno savo taško aplinka.
\end{defn}

\begin{defn}[Uždara aibė]
  Aibė $A$ vadinama uždara aibe, jei $\RSET \setminus A$ yra atvira aibė.
\end{defn}

\begin{note}
  Jei taškas $a \in \{-\infty; +\infty\}$, tai jo aplinka:
  \begin{align*}
    U(\varepsilon; +\infty) &= (\varepsilon; +\infty) \\
    U(\varepsilon; -\infty) &= (-\infty; -\varepsilon)
  \end{align*}
\end{note}

\begin{notation}
  Taško $a (a \in \RSET \cup \{-\infty;+\infty\})$ aplinka žymima:
  $U_{a}$ arba $V_{a}$.
\end{notation}

Šiuose apibrėžimuose laikome, jog $A \subset \RSET$:
\begin{enumerate}
  \item 
    \begin{defn}[Aibės ribinis taškas]
      Taškas $a (a \in \RSET)$ vadinamas aibės $A$ ribiniu tašku, jei 
      $\forall \varepsilon (\varepsilon > 0)$ egzistuoja bent du aibės
      taškai priklausantys $U(\varepsilon; a)$.
    \end{defn}
  \item
    \begin{defn}[Aibės vidinis taškas]
      Taškas $a (a \in \RSET)$ vadinamas vidiniu aibės $A$ tašku, jei 
      $\exists \varepsilon (\varepsilon > 0)$, kad 
      $U(\varepsilon; a) \subset A$.
    \end{defn}
  \item 
    \begin{defn}[Aibės sienos taškas]
      Taškas $a (a \in \RSET)$ vadinamas aibės $A$ sienos tašku, jei 
      $\forall \varepsilon (\varepsilon > 0)$ 
      $U(\varepsilon; a)$ turės tašką iš aibės $A$ ir tašką 
      nepriklausantį $A$.
    \end{defn}
  \item 
    \begin{defn}[Aibės izoliuotas taškas]
      Taškas $a (a \in \RSET)$ vadinamas aibės $A$ izoliuotu tašku, jei
      $a \in A$ ir $\exists \varepsilon (\varepsilon > 0)$, 
      $U(\varepsilon; a)$, kurioje be $a$ nėra kitų aibės taškų.
    \end{defn}
\end{enumerate}
