\chapter{Funkcijos riba}

\section{Taško aplinka, aibė, jos taškai}

\begin{notation}
  Funkcija iš realiųjų skaičių aibės $A$ į visų realiųjų skaičių aibę 
  $\RSET$ žymima:
  \[
  \begin{array}[]{c l}
    f : A \to \RSET, & \text{kur $A$ – funkcijos $f$ apibrėžimo sritis.}
  \end{array}
  \]
\end{notation}

\begin{defn}[$\varepsilon$-aplinka]
  \[
  B(\varepsilon; a) = U(\varepsilon; a) =%
  \{ x \in \RSET : |x - a| < \varepsilon \}
  \]
\end{defn}

\begin{defn}[Taško aplinka]
  Taško $a$ aplinka vadinsime aibę $A$, kuriai:
  \[
  \varepsilon > 0 : \exists U(\varepsilon; a) \subset A
  \]
\end{defn}

\begin{defn}[Atvira aibė]
  Aibė $A$ vadinama atvira aibe, jei ji yra kiekvieno savo taško aplinka.
\end{defn}

\begin{defn}[Uždara aibė]
  Aibė $A$ vadinama uždara aibe, jei $\RSET \setminus A$ yra atvira aibė.
\end{defn}

\begin{note}
  Jei taškas $a \in \{-\infty; +\infty\}$, tai jo aplinka:
  \begin{align*}
    U(\varepsilon; +\infty) &= (\varepsilon; +\infty) \\
    U(\varepsilon; -\infty) &= (-\infty; -\varepsilon)
  \end{align*}
\end{note}

\begin{notation}
  Taško $a (a \in \RSET \cup \{-\infty;+\infty\})$ aplinka žymima:
  $U_{a}$ arba $V_{a}$.
\end{notation}

Šiuose apibrėžimuose laikome, jog $A \subset \RSET$:
\begin{enumerate}
  \item 
    \begin{defn}[Aibės ribinis taškas]
      Taškas $a (a \in \RSET)$ vadinamas aibės $A$ ribiniu tašku, jei 
      $\forall \varepsilon (\varepsilon > 0)$ egzistuoja bent du aibės
      taškai priklausantys $U(\varepsilon; a)$.
    \end{defn}
  \item
    \begin{defn}[Aibės vidinis taškas]
      Taškas $a (a \in \RSET)$ vadinamas vidiniu aibės $A$ tašku, jei 
      $\exists \varepsilon (\varepsilon > 0)$, kad 
      $U(\varepsilon; a) \subset A$.
    \end{defn}
  \item 
    \begin{defn}[Aibės sienos taškas]
      Taškas $a (a \in \RSET)$ vadinamas aibės $A$ sienos tašku, jei 
      $\forall \varepsilon (\varepsilon > 0)$ 
      $U(\varepsilon; a)$ turės tašką iš aibės $A$ ir tašką 
      nepriklausantį $A$.
    \end{defn}
  \item 
    \begin{defn}[Aibės izoliuotas taškas]
      Taškas $a (a \in \RSET)$ vadinamas aibės $A$ izoliuotu tašku, jei
      $a \in A$ ir $\exists \varepsilon (\varepsilon > 0)$, 
      $U(\varepsilon; a)$, kurioje be $a$ nėra kitų aibės taškų.
    \end{defn}
\end{enumerate}

\begin{exmp}
  Panagrinėkime aibę $A = \left\{ \frac{1}{n} : n \in \NSET \right\}% 
  \cup \left[ 10; 20 \right)$:

  \begin{center}
    \begin{tabular}[]{c c c c c}
      Taškas & Ar ribinis & Ar vidinis & Ar sienos & Ar izoliuotas \\
      10 & Taip & Ne & Taip & Ne \\
      20 & Taip & Ne & Taip & Ne \\
      15 & Taip & Taip & Ne & Ne \\
      $\frac{1}{2}$ & Ne & Ne & Taip & Taip \\
      0 & Taip & Ne & Taip & Ne 
    \end{tabular}
  \end{center}
\end{exmp}

\section{Funkcijos riba}

\begin{note}
  Toliau laikome, jog turime funkciją $f : A \to B$ ir kad taškas $a$ 
  yra aibės $A$ ribinis taškas.
\end{note}

\begin{defn}[Funkcijos riba]
  \label{limfed}
  Taškas $b$ vadinamas funkcijos $f$ riba taške $a$, jei:
  \[
  \forall U_b, \exists U_a : f(x) \in U_b,%
  \forall x (x \in U_a \cap A \setminus \{a\})
  \]
\end{defn}

\begin{defn}[Funkcijos riba]
  Taškas $b (b \in \RSET)$ vadinamas funkcijos $f$ riba taške 
  $a (a \in \RSET)$, jei:
  \[
  \forall \varepsilon (\varepsilon > 0), \exists \delta (\delta > 0):%
  |f(x) - b| < \varepsilon,%
  \forall x (x \in A \setminus \{a\} : |x - a| < \delta)
  \]
\end{defn}

\begin{defn}[Funkcijos riba]
  Taškas $b = +\infty$ vadinamas funkcijos $f$ riba taške $a (a \in \RSET)$,
  jei:
  \[
  \forall \varepsilon (\varepsilon > 0), \exists \delta (\delta > 0) :%
  f(x) > \varepsilon,%
  \forall x (x \in A \setminus \{a\} : |x - a| < \delta)
  \]
\end{defn}

TODO: Užrašyti ribos apibrėžimą, kai $b=-\infty \land a=+\infty$, 
$b \in \RSET \land a=-\infty$.

\begin{notation}
  Funkcijos $f(x)$ riba taške $a$, kuri lygi $b$ žymima:
  \[
  \lim _{x \to a} f(x) = b
  \]
\end{notation}

TODO: Užrašyti duotąjį pavyzdį.

\begin{defn}[Funkcijos riba]
  \label{limfs}
  Taškas $b$ vadinamas $f$ riba taške $a$, jei:
  \[
  \forall \left\{ x_{n} \right\}, x_{n} \to a :%
  f(x_{n}) \to b, \text{ kur } x_{n} \in A \text{ ir } x_{n} \neq a \:%
  \forall n (n \in \NSET)%
  \]
\end{defn}

TODO: Užrašyti per praktiką analizuotus pavyzdžius.

TODO: Sutvarkyti teiginio ir jo įrodymo pavyzdį.
\begin{prop}
  \ref{limfed} ir \ref{limfs} funkcijos ribos apibrėžimai yra ekvivalentūs.
  \begin{proof}
    \hfill \\
    \begin{description}
      \item[(\ref{limfed} $\implies$ \ref{limfs})] 
        Sakykime $\lim _{x \to a} f(x) = b$ pagal \ref{limfed} apibrėžimą.

        Panagrinėkime seką $\{x_{n}\}$, kuri artėja į $a$:
        \begin{equation}
          x_{n} \to a, x_{n} \in A, x_{n} \neq a.
          \label{limfp01}
        \end{equation}
        Pagal skaičių sekos ribos apibrėžimą:
        \begin{equation}
          \forall U_{a}, \exists N (N \in \NSET) :%
            x_{n} \in U_{a}, \forall n (n > N).
          \label{limfp02}
        \end{equation}

        Dabar laisvai pasirinkime $U_{b}$. Pagal \ref{limfed} apibrėžimą 
        ir mūsų prielaidą (jog $\lim _{x \to a} f(x) = b$):
        \begin{equation}
          \forall U_{b}, \exists U_{a} :%
            f(x) \in U_{b}, \forall x (x \in U_{a}).
          \label{limfp03}
        \end{equation}

        Kadangi \ref{limfp03} teisinga visiems $x$, kurie priklauso $U_{a}$,
        tai ji teisinga ir visiems $x_{n}$, kurie priklauso $U_{a}$. 
        Todėl iš \ref{limfp03} gauname:
        \begin{equation}
          f(x_{n}) \in U_{b}, \forall x_{n} (x_{n} \in U_{a}).
          \label{limfp04}
        \end{equation}

        Kadangi pagal \ref{limfp03} visoms $U_{b}$ egzistuoja bent viena
        $U_{a}$ ir ji būtinai tenkina \ref{limfp02}, tai iš \ref{limfp04}
        ir \ref{limfp02} gauname:
        \begin{equation}
          f(x_{n}) \in U_{b}, \forall n (n > N),
          \label{limfp05}
        \end{equation}
        o tai ir reiškia, jog $f(x_{n}) \to b$.

      \item[(\ref{limfs} $\implies$ \ref{limfed})]
        Sakykime $\lim _{x \to a} f(x) = b$ pagal \ref{limfs} apibrėžimą.

        Tarkime, jog taškas $b$ nėra riba taške a pagal \ref{limfed}
        apibrėžimą. Tada:
        \begin{equation}
          \exists U_{b}, \forall U_{a} :%
            f(x') \not\in U_{b}, \exists x' (x' \in U_{a}),
          \label{limfp11}
        \end{equation}
        kur $x'$ – kažkoks pasirinktas skaičius.

        Imkime:
        \begin{align*}
          U ^{(n)} _{a} &= \left( a - \frac{1}{n}; a + \frac{1}{n} \right),%
            &a \in \RSET \\
          U ^{(n)} _{+\infty} &= (n; +\infty) \\
          U ^{(n)} _{-\infty} &= (-\infty; -n)
        \end{align*}
        Vietoj $U_{a}$ visur statome $U ^{(n)} _{a}$:
        \begin{align*}
          \exists x_{1} (x_{1} \in U _{a} ^{(1)}) &:%
            f(x_{1}) \not\in U_{b}%
            \text{ (pagal \ref{limfp11})} \\
          \exists x_{2} (x_{2} \in U _{a} ^{(2)}) &:%
            f(x_{2}) \not\in U_{b}%
            \text{ (pagal \ref{limfp11})} \\
          \exists x_{3} (x_{3} \in U _{a} ^{(3)}) &:%
            f(x_{3}) \not\in U_{b}%
            \text{ (pagal \ref{limfp11})} \\
          \cdots
        \end{align*}

        Sukonstravome $\left\{ x_{n} \right\}$, kuri pagal susitraukiančiųjų
        intervalų teoremą $x_{n} \to a$, nes
        \[
        \overbrace{\left(a+\frac{1}{n} - \left(a-\frac{1}{n}\right)\right)%
          }^{U ^{(n)} _{a} ilgis} = \frac{2}{n} \to 0
        \]
        ir
        \[
        a - \frac{1}{n} \leq x_{n} \leq a + \frac{1}{n}.
        \]
        Pagal skaičių sekos ribos apibrėžimą gauname, jog 
        $f(x_{n}) \not\to b$, nes $\exists U_{b}, \forall N:%
        f(x_{n}) \not\in U_{b}, \exists n (n > N)$. Bet tai prieštarauja
        mūsų prielaidai, jog $\lim _{x \to a} f(x) = b$ pagal \ref{limfs}
        apibrėžimą.
    \end{description}
  \end{proof}
\end{prop}
