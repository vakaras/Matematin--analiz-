\documentclass[a4paper]{report}

\usepackage{fontspec}
\usepackage{xltxtra}
\usepackage[lithuanian]{babel}
\usepackage{indentfirst}
\usepackage[]{hyperref}
\usepackage[]{amsmath}
\usepackage{amsthm}
\usepackage{amsfonts}
\usepackage{alltt}
%\usepackage{listingsutf8}

\hypersetup{pdfborder={0 0 0 0}}

\defaultfontfeatures{Mapping=tex-text}

% Aibių žymėjimai.
\newcommand\RSET{\mathop{\mathbb{R}}}
\newcommand\NSET{\mathop{\mathbb{N}}}

% Apibrėžimai, teiginiai, pastabos.
\swapnumbers

\theoremstyle{plain}
\newtheorem{prop}{Tg}

\theoremstyle{definition}
\newtheorem{defn}{Ap.}
\newtheorem{exmp}{Pvz}

\theoremstyle{remark}
\newtheorem*{note}{Pastaba}

\newtheoremstyle{notation}{3pt}{3pt}{}{}{\itshape}{:}{.5em}{}
\theoremstyle{notation}
\newtheorem*{notation}{Žymėjimas}

\title{%
Almanto Juozulyno\\
Matematinės analizės paskaitų konspektas}

\author{}
\date{\today}

\begin{document}

\maketitle
\bigskip
\tableofcontents
\chapter{Įžanga}

\section{Vertinimas}

Galutinis pažymys sudaromas susumavus visus taškus:
\begin{itemize}
  \item kontrolinis – 10 taškų;
  \item egzaminas – 20 taškų ir
  \item papildoma užduotis – 4 taškai.
\end{itemize}

\begin{note}
  Egzamine gali būti klausimai iš viso kurso.
\end{note}

\section{Temos}

Kurso metu nagrinėjamos temos:
\begin{enumerate}
  \item Funkcijos riba ir tolydumas.
  \item Funkcijos išvestinė (Teiloro formulė, funkcijos tyrimas, 
    išvestinės apibrėžimas).
  \item Apibrėžtinis integralas (Rymano, Lebego, nelabai konkretaus
    pavadinimo integralai).
  \item Netiesioginis integralas (integravimo kintamojo pakeitimo formulė,
    integravimo dalimis formulė, Niutono-Leibnico formulė.)
\end{enumerate}
                 % Įžanginė.
\chapter{Funkcijos riba}

\begin{notation}
  Funkcija iš realiųjų skaičių aibės $A$ į visų realiųjų skaičių aibę 
  $\RSET$ žymima:
  \[
  \begin{array}[]{c l}
    f : A \to \RSET, & \text{kur $A$ – funkcijos $f$ apibrėžimo sritis.}
  \end{array}
  \]
\end{notation}

\begin{defn}[$\varepsilon$-aplinka]
  \[
  B(\varepsilon; a) = U(\varepsilon; a) =%
  \{ x \in \RSET : |x - a| < \varepsilon \}
  \]
\end{defn}

\begin{defn}[Taško aplinka]
  Taško $a$ aplinka vadinsime aibę $A$, kuriai:
  \[
  \varepsilon > 0 : \exists U(\varepsilon; a) \subset A
  \]
\end{defn}

\begin{defn}[Atvira aibė]
  Aibė $A$ vadinama atvira aibe, jei ji yra kiekvieno savo taško aplinka.
\end{defn}

\begin{defn}[Uždara aibė]
  Aibė $A$ vadinama uždara aibe, jei $\RSET \setminus A$ yra atvira aibė.
\end{defn}

\begin{note}
  Jei taškas $a \in \{-\infty; +\infty\}$, tai jo aplinka:
  \begin{align*}
    U(\varepsilon; +\infty) &= (\varepsilon; +\infty) \\
    U(\varepsilon; -\infty) &= (-\infty; -\varepsilon)
  \end{align*}
\end{note}

\begin{notation}
  Taško $a (a \in \RSET \cup \{-\infty;+\infty\})$ aplinka žymima:
  $U_{a}$ arba $V_{a}$.
\end{notation}

Šiuose apibrėžimuose laikome, jog $A \subset \RSET$:
\begin{enumerate}
  \item 
    \begin{defn}[Aibės ribinis taškas]
      Taškas $a (a \in \RSET)$ vadinamas aibės $A$ ribiniu tašku, jei 
      $\forall \varepsilon (\varepsilon > 0)$ egzistuoja bent du aibės
      taškai priklausantys $U(\varepsilon; a)$.
    \end{defn}
  \item
    \begin{defn}[Aibės vidinis taškas]
      Taškas $a (a \in \RSET)$ vadinamas vidiniu aibės $A$ tašku, jei 
      $\exists \varepsilon (\varepsilon > 0)$, kad 
      $U(\varepsilon; a) \subset A$.
    \end{defn}
  \item 
    \begin{defn}[Aibės sienos taškas]
      Taškas $a (a \in \RSET)$ vadinamas aibės $A$ sienos tašku, jei 
      $\forall \varepsilon (\varepsilon > 0)$ 
      $U(\varepsilon; a)$ turės tašką iš aibės $A$ ir tašką 
      nepriklausantį $A$.
    \end{defn}
  \item 
    \begin{defn}[Aibės izoliuotas taškas]
      Taškas $a (a \in \RSET)$ vadinamas aibės $A$ izoliuotu tašku, jei
      $a \in A$ ir $\exists \varepsilon (\varepsilon > 0)$, 
      $U(\varepsilon; a)$, kurioje be $a$ nėra kitų aibės taškų.
    \end{defn}
\end{enumerate}

\begin{exmp}
  Panagrinėkime aibę $A = \left\{ \frac{1}{n} : n \in \NSET \right\}% 
  \cup \left[ 10; 20 \right)$:

  \begin{center}
    \begin{tabular}[]{c c c c c}
      Taškas & Ar ribinis & Ar vidinis & Ar sienos & Ar izoliuotas \\
      10 & Taip & Ne & Taip & Ne \\
      20 & Taip & Ne & Taip & Ne \\
      15 & Taip & Taip & Ne & Ne \\
      $\frac{1}{2}$ & Ne & Ne & Taip & Taip \\
      0 & Taip & Ne & Taip & Ne 
    \end{tabular}
  \end{center}
\end{exmp}

Toliau laikome, jog turime funkciją $f : A \to B$ ir kad taškas $a$ 
yra aibės $A$ ribinis taškas.

\begin{defn}[Funkcijos riba]
  \label{limfed}
  Taškas $b$ vadinamas funkcijos $f$ riba taške $a$, jei:
  \[
  \forall U_b, \exists U_a : f(x) \in U_b,%
  \forall x (x \in U_a \cap A \setminus \{a\})
  \]
\end{defn}

\begin{defn}[Funkcijos riba]
  Taškas $b (b \in \RSET)$ vadinamas funkcijos $f$ riba taške 
  $a (a \in \RSET)$, jei:
  \[
  \forall \varepsilon (\varepsilon > 0), \exists \delta (\delta > 0):%
  |f(x) - b| < \varepsilon,%
  \forall x (x \in A \setminus \{a\} : |x - a| < \delta)
  \]
\end{defn}

\begin{defn}[Funkcijos riba]
  Taškas $b = +\infty$ vadinamas funkcijos $f$ riba taške $a (a \in \RSET)$,
  jei:
  \[
  \forall \varepsilon (\varepsilon > 0), \exists \delta (\delta > 0) :%
  f(x) > \varepsilon,%
  \forall x (x \in A \setminus \{a\} : |x - a| < \delta)
  \]
\end{defn}

TODO: Užrašyti ribos apibrėžimą, kai $b=-\infty \land a=+\infty$, 
$b \in \RSET \land a=-\infty$.

\begin{notation}
  Funkcijos $f(x)$ riba taške $a$, kuri lygi $b$ žymima:
  \[
  \lim _{x \to a} f(x) = b
  \]
\end{notation}

TODO: Užrašyti duotąjį pavyzdį.

\begin{defn}[Funkcijos riba]
  \label{limfs}
  Taškas $b$ vadinamas $f$ riba taške $a$, jei:
  \[
  \forall \left\{ x_{n} \right\}, x_{n} \to a :%
  f(x_{n}) \to b, \text{ kur } x_{n} \in A \text{ ir } x_{n} \neq a \:%
  \forall n (n \in \NSET)%
  \]
\end{defn}

TODO: Užrašyti per praktiką analizuotus pavyzdžius.

TODO: Sutvarkyti teiginio ir jo įrodymo pavyzdį.
\begin{prop}
  \ref{limfed} ir \ref{limfs} funkcijos ribos apibrėžimai yra ekvivalentūs.
  \begin{proof}
    \hfill \\
    \begin{description}
      \item[(\ref{limfed} $\implies$ \ref{limfs})] 
        Sakykime $\lim _{x \to a} f(x) = b$ pagal \ref{limfed} apibrėžimą.
      \item[(\ref{limfs} $\implies$ \ref{limfed})]
        Sakykime $\lim _{x \to a} f(x) = b$ pagal \ref{limfs} apibrėžimą.
    \end{description}
  \end{proof}
\end{prop}
                 % Funkcijos riba.
\begin{prop}
  Jei $\lim _{x \to a} f(x) = b$ ir $\lim _{x \to a} f(x) = c$, tai $b = c$.
  \begin{proof}
    Tarkime priešinai $b \neq c$. Tegu $b < c$.

    Pagal \ref{limfed} apibrėžimą:
    \begin{align*}
      \lim _{x \to a} f(x) = b \iff&% 
        \forall U_{b}, \exists U_{a} :%
        f(x) \in U_{b}, \forall x(x \in U_{a}) \\
      \lim _{x \to a} f(x) = c \iff&% 
        \forall U_{c}, \exists U'_{a} :%
        f(x) \in U_{c}, \forall x(x \in U'_{a})
    \end{align*}

    Pastebėkime, kad $U_{a} \cap U'_{a}$ bus taško $a$ aplinka (taško 
    aplinka visada yra netuščia aibė).
    Fiksuojame tokias $U_{b}$ ir $U_{c}$, kurioms 
    $U_{b} \cap U_{c} = \emptyset$. Bet 
    $\forall x (x \in U_{a} \cap U'_{a}) \implies% 
      f(x) \in U_{b} \land f(x) \in U_{c}$, gavome prieštarą.
  \end{proof}
\end{prop}

\begin{prop}
  Jei $f(x) \leq g(x), \forall x (x \in U_{a})$ ir $f$ bei $g$ turi ribą
  taške $a$, tai $\lim _{x \to a} f(x) \leq \lim _{x \to a} g(x)$.
  %TODO: Įrodyti!
\end{prop}

\begin{prop}
  Jei $\lim _{x \to a} f(x) = b < c$, tai 
  $\exists U_{a} : f(x) < c, \forall x (x \in U_{a})$.
  %TODO: Įrodyti!
\end{prop}

\begin{prop}
  Jei $h(x) \leq f(x) \leq g(x), \forall x(x \in A)$ ir 
  $\lim _{x \to a} h(x) = \lim _{x \to a} g(x) = b$, tai
  $\lim _{x \to a} f(x) = b$.
  %TODO: Įrodyti!
\end{prop}

\begin{prop}
  Jei $f$ yra monotoniška intervale $I$ ir $a \in I$, tai 
  $\exists \lim _{x \to a} f(x)$. Jei papildomai žinome, kad $f$ yra
  aprėžta intervale $I$, tai $\lim _{x \to a} f(x)$ yra baigtinė.
  %TODO: Įrodyti!
\end{prop}

\begin{prop}
  Jei $f$ ir $g$ turi baigtines ribas taške $a$ ir 
  $\alpha, \beta \in \RSET$, tai 
  \begin{enumerate}
    \item \[
      \lim _{x \to a} (\alpha f(x) + \beta g(x)) =%
        \alpha \lim _{x \to a} f(x) + \beta \lim _{x \to a} g(x);
      \]
    \item \[
      \lim _{x \to a} (f(x)g(x)) =%
        \lim _{x \to a} f(x) \lim _{x \to a} g(x);
      \]
    \item Jei 
      $g(x) \neq 0, \forall x (x \in A), \lim _{x \to a} g(x) \neq 0$, tai
      \[
      \lim _{x \to a} \frac{f(x)}{g(x)} =%
        \frac{\lim _{x \to a} f(x)}{\lim _{x \to a} g(x)}.
      \]
  %TODO: Įrodyti!
  \end{enumerate}
\end{prop}

\begin{defn}[Funkcijos riba iš dešinės]
  Taškas $b$ vadinamas funkcijos $f$ riba taške $a$ iš dešinės, jei:
  \[
  \forall U_{b}, \exists U_{a} :%
    f(x) \in U_{b}, \forall x (x \in U_{a} \cap (a; +\infty)).
  \]
  \begin{notation} 
    \[
    \lim _{x \to a^{+}} f(x) = b
    \]
  \end{notation}
\end{defn}

\begin{defn}[Funkcijos riba iš kairės]
  Taškas $b$ vadinamas funkcijos $f$ riba taške $a$ iš kairės, jei:
  \[
  \forall U_{b}, \exists U_{a} :%
    f(x) \in U_{b}, \forall x (x \in U_{a} \cap (-\infty; a)).
  \]
  \begin{notation} 
    \[
    \lim _{x \to a^{-}} f(x) = b
    \]
  \end{notation}
\end{defn}

TODO: Įkelti per praktiką darytus pavyzdžius.
                 % Funkcijos riba.
\chapter{Pavyzdžiai}

Kadangi praktiškai tą patį su TeX sistema galima surinkti įvairiais
būdais, tai šiame faile turėtų būti laikomi gražiausi/patogiausi 
pavyzdžiai.

Čia turėtų būti kažkoks prasmingas tekstas.

\begin{defn}[Apibrėžimo pavyzdys]
  Čia turėtų būti apibrėžimas.
\end{defn}

\begin{exmp}[Pavyzdys]
  Čia turėtų būti pavyzdys.
\end{exmp}

\begin{note}
  Pastabos pavyzdys.
\end{note}

\begin{prop}
  Teiginio pavyzdys.
  \begin{proof}
    Teiginio įrodymas.
  \end{proof}
\end{prop}

\begin{notation}
  (Žymėjimo pavyzdys.)
  Matematinė formulė, su paaiškintais kintamaisiais:
  \[
  \begin{array}[]{c l}
    f : A \to \RSET, & \text{kur $A$ – funkcijos $f$ apibrėžimo sritis.}
  \end{array}
  \]
\end{notation}

Kelios matematinės formulės, išlygiuotos ties $=$:
\begin{align*}
  U(\varepsilon; +\infty) &= (\varepsilon; +\infty) \\
  U(\varepsilon; -\infty) &= (-\infty; -\varepsilon)
\end{align*}
                         % Įvairus TeX pavyzdžiai.
\appendix
\chapter{Pagalbinės formulės uždaviniams spręsti}

\section{Ribų skaičiavimo formulės}

\begin{align} 
%
  &\lim _{x \to 0} \frac{\sin x}{x} = 1
  \label{f_lim_sin} \\
%
  &\lim _{x \to 0} \left( 1 + x \alpha \right)^{\frac{1}{x}} = e^{\alpha}
  \label{f_lim_exp} \\
%
  &\lim _{x \to 0} \frac{a^x - 1}{x} = \ln a, 
  & \text{ kur } a > 0, a \neq 1
  \label{f_lim_ep} \\
% 
  &\lim _{x \to 0} \frac{(1 + x)^r -1}{x} = r, & \text{ kur } r \in \RSET
  \label{f_lim_lp} \\
%
  &\lim _{x \to 0} \frac{\log _{a} (1 + x)}{x} = \log _{a} e, 
  & \text{ kur } a > 0, a \neq 1
  \label{f_lim_log} \\
%
  &\lim _{x \to +\infty} \frac{a^x}{x^r} = +\infty, 
  & \text{ kur } a > 1, r \in \RSET
  \label{f_lim_rl} \\
% 
  &\lim _{x \to +\infty} \frac{\log _{a} x}{x^r} = 0,
  & \text{ kur } a > 1, r > 0
  \label{f_lim_llb} \\
%
  &\lim _{x \to 0} x^p \log _{a} x = 0, & \text{ kur } a > 1, p > 0
  \label{f_lim_lln}
%
\end{align}

\section{Trigonometrinės tapatybės}

\begin{align}
%
  & \sin ^{2} \alpha + \cos ^{2} \alpha = 1
  \label{f_tri_kvsum} \\
%
  & \sin \alpha - \sin \beta = 2 
    \sin \left( \frac{\alpha - \beta}{2} \right)
    \cos \left( \frac{\alpha + \beta}{2} \right)
  \label{f_tri_sinsk} \\
%
  & \cos 2 \alpha = \cos ^{2} \alpha - \sin ^{2} \alpha
  \label{f_tri_dkcos} \\
%
  & \sin 2 \alpha = 2 \sin \alpha \cos \alpha
  \label{f_tri_dksin} 
%
\end{align}
                    % Pagalbinės formulės uždaviniams
                                        % spręsti.

\end{document}
