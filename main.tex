\documentclass[a4paper]{report}

\usepackage{fontspec}
\usepackage{xltxtra}
\usepackage[lithuanian]{babel}
\usepackage{indentfirst}
\usepackage[]{hyperref}
\usepackage[]{amsmath}
\usepackage{amsthm}
\usepackage{amsfonts}
\usepackage{alltt}
%\usepackage{listingsutf8}

\hypersetup{pdfborder={0 0 0 0}}

\defaultfontfeatures{Mapping=tex-text}

% Aibių žymėjimai.
\newcommand\RSET{\mathop{\mathbb{R}}}
\newcommand\NSET{\mathop{\mathbb{N}}}

% Apibrėžimai, teiginiai, pastabos.
\swapnumbers

\theoremstyle{plain}
\newtheorem{prop}{Tg}

\theoremstyle{definition}
\newtheorem{defn}{Ap.}
\newtheorem{exmp}{Pvz}

\theoremstyle{remark}
\newtheorem*{note}{Pastaba}

\newtheoremstyle{notation}{3pt}{3pt}{}{}{\itshape}{:}{.5em}{}
\theoremstyle{notation}
\newtheorem*{notation}{Žymėjimas}

\title{%
Almanto Juozulyno\\
Matematinės analizės paskaitų konspektas}

\author{}
\date{\today}

\begin{document}

\maketitle
\bigskip
\tableofcontents
\chapter{Įžanga}

Čia turėtų būti kažkoks prasmingas tekstas.

\begin{defn}[Apibrėžimo pavyzdys]
  Čia turėtų būti apibrėžimas.
\end{defn}

\begin{exmp}[Pavyzdys]
  Čia turėtų būti pavyzdys.
\end{exmp}

\begin{note}
  Pastabos pavyzdys.
\end{note}

\begin{prop}
  Teiginio pavyzdys.
  \begin{proof}
    Teiginio įrodymas.
  \end{proof}
\end{prop}
                 % Įžanginė.
\chapter{Funkcijos riba}

\begin{notation}
  Funkcija iš realiųjų skaičių aibės $A$ į visų realiųjų skaičių aibę 
  $\RSET$ žymima:
  \[
  \begin{array}[]{c l}
    f : A \to \RSET, & \text{kur $A$ – funkcijos $f$ apibrėžimo sritis.}
  \end{array}
  \]
\end{notation}

\begin{defn}[$\varepsilon$-aplinka]
  \[
  B(\varepsilon; a) = U(\varepsilon; a) =%
  \{ x \in \RSET : |x - a| < \varepsilon \}
  \]
\end{defn}

\begin{defn}[Taško aplinka]
  Taško $a$ aplinka vadinsime aibę $A$, kuriai:
  \[
  \varepsilon > 0 : \exists U(\varepsilon; a) \subset A
  \]
\end{defn}

\begin{defn}[Atvira aibė]
  Aibė $A$ vadinama atvira aibe, jei ji yra kiekvieno savo taško aplinka.
\end{defn}

\begin{defn}[Uždara aibė]
  Aibė $A$ vadinama uždara aibe, jei $\RSET \setminus A$ yra atvira aibė.
\end{defn}

\begin{note}
  Jei taškas $a \in \{-\infty; +\infty\}$, tai jo aplinka:
  \begin{align*}
    U(\varepsilon; +\infty) &= (\varepsilon; +\infty) \\
    U(\varepsilon; -\infty) &= (-\infty; -\varepsilon)
  \end{align*}
\end{note}

\begin{notation}
  Taško $a (a \in \RSET \cup \{-\infty;+\infty\})$ aplinka žymima:
  $U_{a}$ arba $V_{a}$.
\end{notation}

Šiuose apibrėžimuose laikome, jog $A \subset \RSET$:
\begin{enumerate}
  \item 
    \begin{defn}[Aibės ribinis taškas]
      Taškas $a (a \in \RSET)$ vadinamas aibės $A$ ribiniu tašku, jei 
      $\forall \varepsilon (\varepsilon > 0)$ egzistuoja bent du aibės
      taškai priklausantys $U(\varepsilon; a)$.
    \end{defn}
  \item
    \begin{defn}[Aibės vidinis taškas]
      Taškas $a (a \in \RSET)$ vadinamas vidiniu aibės $A$ tašku, jei 
      $\exists \varepsilon (\varepsilon > 0)$, kad 
      $U(\varepsilon; a) \subset A$.
    \end{defn}
  \item 
    \begin{defn}[Aibės sienos taškas]
      Taškas $a (a \in \RSET)$ vadinamas aibės $A$ sienos tašku, jei 
      $\forall \varepsilon (\varepsilon > 0)$ 
      $U(\varepsilon; a)$ turės tašką iš aibės $A$ ir tašką 
      nepriklausantį $A$.
    \end{defn}
  \item 
    \begin{defn}[Aibės izoliuotas taškas]
      Taškas $a (a \in \RSET)$ vadinamas aibės $A$ izoliuotu tašku, jei
      $a \in A$ ir $\exists \varepsilon (\varepsilon > 0)$, 
      $U(\varepsilon; a)$, kurioje be $a$ nėra kitų aibės taškų.
    \end{defn}
\end{enumerate}
                 % Funkcijos riba.
\begin{prop}
  Jei $\lim_{x \to a} f(x) = b$ ir $\lim_{x \to a} f(x) = c$, tai $b = c$.
  \begin{proof}
    Tarkime priešinai $b \neq c$. Tegu $b < c$.

    Pagal \ref{limfed} apibrėžimą:
    \begin{align*}
      \lim_{x \to a} f(x) = b \iff&% 
        \forall U_{b}, \exists U_{a} :%
        f(x) \in U_{b}, \forall x(x \in U_{a}) \\
      \lim_{x \to a} f(x) = c \iff&% 
        \forall U_{c}, \exists U'_{a} :%
        f(x) \in U_{c}, \forall x(x \in U'_{a})
    \end{align*}

    Pastebėkime, kad $U_{a} \cap U'_{a}$ bus taško $a$ aplinka (taško 
    aplinka visada yra netuščia aibė).
    Fiksuojame tokias $U_{b}$ ir $U_{c}$, kurioms 
    $U_{b} \cap U_{c} = \emptyset$. Bet 
    $\forall x (x \in U_{a} \cap U'_{a}) \implies% 
      f(x) \in U_{b} \land f(x) \in U_{c}$, gavome prieštarą.
  \end{proof}
\end{prop}

\begin{prop}
  Jei $f(x) \leq g(x), \forall x (x \in U_{a})$ ir $f$ bei $g$ turi ribą
  taške $a$, tai $\lim_{x \to a} f(x) \leq \lim_{x \to a} g(x)$.
  %TODO: Įrodyti!
\end{prop}

\begin{prop}
  Jei $\lim_{x \to a} f(x) = b < c$, tai 
  $\exists U_{a} : f(x) < c, \forall x (x \in U_{a})$.
  %TODO: Įrodyti!
\end{prop}

\begin{prop}
  Jei $h(x) \leq f(x) \leq g(x), \forall x(x \in A)$ ir 
  $\lim_{x \to a} h(x) = \lim_{x \to a} g(x) = b$, tai
  $\lim_{x \to a} f(x) = b$.
  %TODO: Įrodyti!
\end{prop}

\begin{prop}
  Jei $f$ yra monotoniška intervale $I$ ir $a \in I$, tai 
  $\exists \lim_{x \to a} f(x)$. Jei papildomai žinome, kad $f$ yra
  aprėžta intervale $I$, tai $\lim_{x \to a} f(x)$ yra baigtinė.
  %TODO: Įrodyti!
\end{prop}

\begin{prop}
  Jei $f$ ir $g$ turi baigtines ribas taške $a$ ir 
  $\alpha, \beta \in \RSET$, tai 
  \begin{enumerate}
    \item \[
      \lim_{x \to a} (\alpha f(x) + \beta g(x)) =%
        \alpha \lim_{x \to a} f(x) + \beta \lim_{x \to a} g(x);
      \]
    \item \[
      \lim_{x \to a} (f(x)g(x)) =%
        \lim_{x \to a} f(x) \lim_{x \to a} g(x);
      \]
    \item Jei 
      $g(x) \neq 0, \forall x (x \in A), \lim_{x \to a} g(x) \neq 0$, tai
      \begin{equation*}
        \lim_{x \to a} \frac{f(x)}{g(x)} =
          \frac{\lim_{x \to a} f(x)}{\lim_{x \to a} g(x)}.
      \end{equation*}
  %TODO: Įrodyti!
  \end{enumerate}
\end{prop}

\begin{defn}[Funkcijos riba iš dešinės]
  Taškas $b$ vadinamas funkcijos $f$ riba taške $a$ iš dešinės, jei:
  \[
  \forall U_{b}, \exists U_{a} :%
    f(x) \in U_{b}, \forall x (x \in U_{a} \cap (a; +\infty)).
  \]
  \begin{notation} 
    \[
    \lim_{x \to a^{+}} f(x) = b
    \]
  \end{notation}
\end{defn}

\begin{defn}[Funkcijos riba iš kairės]
  Taškas $b$ vadinamas funkcijos $f$ riba taške $a$ iš kairės, jei:
  \[
  \forall U_{b}, \exists U_{a} :%
    f(x) \in U_{b}, \forall x (x \in U_{a} \cap (-\infty; a)).
  \]
  \begin{notation} 
    \[
    \lim_{x \to a^{-}} f(x) = b
    \]
  \end{notation}
\end{defn}

TODO: Įkelti per praktiką darytus pavyzdžius.
                 % Funkcijos riba.
\chapter{Pavyzdžiai}

Kadangi praktiškai tą patį su TeX sistema galima surinkti įvairiais
būdais, tai šiame faile turėtų būti laikomi gražiausi/patogiausi 
pavyzdžiai.

Čia turėtų būti kažkoks prasmingas tekstas.

\begin{defn}[Apibrėžimo pavyzdys]
  Čia turėtų būti apibrėžimas.
\end{defn}

\begin{exmp}[Pavyzdys]
  Čia turėtų būti pavyzdys.
\end{exmp}

\begin{note}
  Pastabos pavyzdys.
\end{note}

\begin{prop}
  Teiginio pavyzdys.
  \begin{proof}
    Teiginio įrodymas.
  \end{proof}
\end{prop}

\begin{notation}
  (Žymėjimo pavyzdys.)
  Matematinė formulė, su paaiškintais kintamaisiais:
  \[
  \begin{array}[]{c l}
    f : A \to \RSET, & \text{kur $A$ – funkcijos $f$ apibrėžimo sritis.}
  \end{array}
  \]
\end{notation}

Kelios matematinės formulės, išlygiuotos ties $=$:
\begin{align*}
  U(\varepsilon; +\infty) &= (\varepsilon; +\infty) \\
  U(\varepsilon; -\infty) &= (-\infty; -\varepsilon)
\end{align*}
                         % Įvairus TeX pavyzdžiai.
\appendix
\chapter{Pagalbinės formulės uždaviniams spręsti}

\section{Ribų skaičiavimo formulės}

\begin{align} 
%
  &\lim _{x \to 0} \frac{\sin x}{x} = 1
  \label{f_lim_sin} \\
%
  &\lim _{x \to 0} \left( 1 + x \alpha \right)^{\frac{1}{x}} = e^{\alpha}
  \label{f_lim_exp} \\
%
  &\lim _{x \to 0} \frac{a^x - 1}{x} = \ln a, 
  & \text{ kur } a > 0, a \not{=}1
  \label{f_lim_ep} \\
% 
  &\lim _{x \to 0} \frac{(1 + x)^r -1}{x} = r, & \text{ kur } r \in \RSET
  \label{f_lim_lp} \\
%
  &\lim _{x \to 0} \frac{\log _{a} (1 + x)}{x} = \log _{a} e, 
  & \text{ kur } a > 0, a \not{=} 1
  \label{f_lim_log} \\
%
  &\lim _{x \to +\infty} \frac{a^x}{x^r} = +\infty, 
  & \text{ kur } a > 1, r \in \RSET
  \label{f_lim_rl} \\
% 
  &\lim _{x \to +\infty} \frac{\log _{a} x}{x^r} = 0,
  & \text{ kur } a > 1, r > 0
  \label{f_lim_llb} \\
%
  &\lim _{x \to 0} x^p \log _{a} x = 0, & \text{ kur } a > 1, p > 0
  \label{f_lim_lln}
%
\end{align}

\section{Trigonometrinės tapatybės}

\begin{align}
%
  & \sin ^{2} \alpha + \cos ^{2} \alpha = 1
  \label{f_tri_kvsum} \\
%
  & \sin \alpha - \sin \beta = 2 
    \sin \left( \frac{\alpha - \beta}{2} \right)
    \cos \left( \frac{\alpha + \beta}{2} \right)
  \label{f_tri_sinsk} \\
%
  & \cos 2 \alpha = \cos ^{2} \alpha - \sin ^{2} \alpha
  \label{f_tri_dkcos} \\
%
  & \sin 2 \alpha = 2 \sin \alpha \cos \alpha
  \label{f_tri_dksin} 
%
\end{align}
                    % Pagalbinės formulės uždaviniams
                                        % spręsti.

\end{document}
