\begin{prop}
  (Monotoniškumo kriterijus) Tegu $f: I \to \RSET$ (I – intervalas) yra
  diferencijuojama visame $I$. Tada:
  \begin{align*}
    \text{Funkcija } f \text{ intervale } I \text{ didėja } 
    &\iff f'(x) \geq 0, \forall x (x \in I) \\
    \text{Funkcija } f \text{ intervale } I \text{ mažėja } 
    &\iff f'(x) \leq 0, \forall x (x \in I)
  \end{align*}.

  \begin{proof}
    \begin{description}
      \item[Būtinumas] Tegu $f$ didėja intervale $I$. Tada reikia įrodyti,
        kad $f'(x) \geq 0, \forall x (x \in I)$. 
        
        Fiksuojame $x_{0} (x_{0} \in I)$.

        Jei $x > x_{0}$, tai
        \begin{equation}
          \frac{\overbrace{f(x) - f(x_{0})}^{\geq 0}}{
            \underbrace{x - x_{0}}_{\geq 0}} 
          \implies \geq 0
          \label{_monot_01}
        \end{equation}

        Jei $x < x_{0}$, tai
        \begin{equation}
          \frac{\overbrace{f(x) - f(x_{0})}^{\leq 0}}{
            \underbrace{x - x_{0}}_{\leq 0}}
          \implies \geq 0
          \label{_monot_02}
        \end{equation}

        Kai $x \to x_{0}$, tai 
        \begin{equation}
          \lim _{x \to x_{0}} \frac{f(x) - f(x_{0})}{x - x_{0}}
          \underbrace{\geq}_{
            \mathclap{\text{Seka iš \ref{_monot_01} ir \ref{_monot_02}}}}
          0
          \label{_monot_03}
        \end{equation}

        Iš \ref{_monot_03}: 
        $f'(x_{0}) \geq 0, \forall x_{0} (x_{0} \in I).$
      \item[Pakankamumas] Tarkime, kad 
        $f'(x) \geq 0, \forall x (x \in I)$. Reikia įrodyti, kad 
        $f$ didėja intervale $I$.

        Imame du taškus 
        $x_{1}, x_{2} (x_{1},x_{2} \in I \text{ir} x_{1} < x_{2})$.
        Aišku, kad $[x_{1};x_{2}] \subset I$ ir $[x_{1}; x_{2}]$ –
        $f$ tolydi, o $(x_{1}; x_{2})$ – diferencijuojama.

        Tada iš Lagranžo teoremos:
        \begin{equation*}
          \exists c (c \in (x_{1};x_{2})) : 
            f(x_{2}) - f(x_{1}) = 
            \underbrace{f'(c)}_{\geq 0}
            \underbrace{(x_{2} - x_{1})}_{\geq 0}.
        \end{equation*}
        Kadangi dešinė pusė $\geq 0$, tai ir kairė pusė:
        \begin{align*}
          f(x_{2}) - f(x_{1}) &\geq 0 \\
          f(x_{2}) &\geq f(x_{1}), \forall x_{1}, x_{2} 
            (x_{1}, x_{2} \in I \text{ ir } x_{2} > x_{1})
        \end{align*}.
        Gavome, kad $f$ didėja visame intervale $I$.

    \end{description}
  \end{proof}
\end{prop}

\begin{prop}
  (Liopitalio taisyklė) Tarkime, jog išpildytos šios salygos:
  \begin{enumerate}
    \item $\exists \lim _{x \to a} \frac{f'(x)}{g'(x)}$
    \item $f(a^{+}) = g(a^{+}) = 0$ arba 
      $g(a^{+}) = f(a^{+}) = \pm \infty$.
    \item $g'(x) \neq 0, \forall x (x \in A)$.
  \end{enumerate}

  Tada teisinga:
  \begin{equation*}
    \lim _{x \to a^{+}} \frac{f(x)}{g(x)} = 
    \lim _{x \to a^{+}} \frac{f'(x)}{g'(x)}.
  \end{equation*}
\end{prop}

\begin{note}
  Liopitalio taisyklė galioja taip pat, kai $x \to a^{-}$ ir kai 
  $x \to a$.
\end{note}

\begin{exmp}
  \begin{align}
    \lim _{x \to +\infty} \frac{\ln x}{x}
    &\underbrace{=}_{\mathclap{
      \text{Darome prielaidą, jog pirma Liopitalio taisyklės 
      prielaida yra tenkinama.}}}
    \lim _{x \to +\infty} \frac{(\ln x)'}{(x)'}
    \label{_liopital_exmp_01} \\
    &= \lim _{x \to +\infty} \frac{\frac{1}{x}}{1} \\
    &= 0
    \label{_liopital_exmp_02}
  \end{align}

  Kadangi suskaičiavome ribą, vadinasi ji egzistuoja ir todėl mūsų
  \ref{_liopital_exmp_01} padaryta prielada yra teisinga.
\end{exmp}

\section{Aukštesnių eilių išvestinės}

\begin{notation}
  \begin{align*}
    f'(x) &– \text{ pirmos eilės išvestinė.} \\
    f''(x) &– \text{ antros eilės išvestinė.} \\
    f'''(x) &– \text{ trečios eilės išvestinė.} \\
    f^{(n)}(x) &– \text{ n-osios eilės išvestinė.}
  \end{align*}
\end{notation}

\begin{defn}[n-osios eilės išvestinė]
  \begin{align*}
    f''(x) &:= (f'(x))' \\
    f^{(n)}(x) &:= (f^{(n-1)}(x))'
  \end{align*}
\end{defn}

\begin{defn}[n-osios eilės diferencialas]
  n-osios eilės diferencialas atitinkantis pokytį $h$:
  \begin{equation*}
    d^{n}_{h} f(x) := d_{h} (d^{n-1}_{h} f(x))
  \end{equation*}
\end{defn}

\begin{exmp}
  \begin{equation*}
    f(x) = 
    \begin{cases}
      0, & x = 0 \\
      x^{3} \sin \left( \frac{1}{x} \right), & x \neq 0
    \end{cases}
  \end{equation*}

  \begin{align*}
    f'(x) 
    &= \left( x^{3} \sin \frac{1}{x} \right)' \\
    &= 3 x^{2} \sin \frac{1}{x} +   
      x^{3} \cos \frac{1}{x} \left( \frac{1}{-x^{2}} \right) \\
    &= 3 x^{2} \sin \frac{1}{x} - x \cos \frac{1}{x}
  \end{align*}

  \begin{align*}
    f'(0)
    &= \lim _{x \to 0} \frac{f(x) - f(0)}{x - 0} \\
    &= \lim _{x \to 0} \frac{x^{3} \sin \frac{1}{x}}{x} \\
    &= \lim _{x \to 0} x^{2} \sin \frac{1}{x} \\
    &= 0
  \end{align*}

  \begin{align*}
    f''(0)
    &= (f'(0))' \\
    &= \lim _{x \to 0} \frac{f'(x) - \overbrace{f'(0)}^{= 0}}{x - 0} \\
    &= \lim _{x \to 0} 
      \frac{3 x^{2} \sin \frac{1}{x} - x \cos \frac{1}{x} - 0}{x} \\
    &= \lim _{x \to 0} 
      \left( 
        \underbrace{3 x \sin \frac{1}{x}}_{\to 0} - 
        \underbrace{\cos \frac{1}{x}}_{\mathclap{\text{neturi išvestinės}}} 
      \right)
  \end{align*}

  Išvada: $\nexists f''(0)$.
\end{exmp}

\section{Teiloro formulė}

\begin{align*}
  P(x) &= a_{0} + a_{1}(x - x_{0}) + a_{2}(x - x_{0})^{2} + \cdots
    + a_{n}(x - x_{0})^{n}, & a_{n} \neq 0 \\
  P(x_{0}) &= a_{0} \\
  P'(x) &= a_{1} + 2a_{2}(x - x_{0}) + 3a_{3}(x - x_{0})^{2} + \cdots
    + n a_{n}(x - x_{0})^{n-1} \\
  P'(x_{0}) &= a_{1} \\
  P''(x) &= 2 a_{2} + 2 \cdot 3 a_{3}(x - x_{0}) + \cdots 
    + n(n - 1)a_{n}(x - x_{0})^{n - 2} \\
  P''(x_{0}) &= 2 a_{2} \implies a_{2} = \frac{P''(x_{0})}{2} \\
  P'''(x) &= 2 \cdot 3 a_{3} + 2 \cdot 3 \cdot 4 a_{4} (x - x_{0}) + \cdots
    + n(n - 1)(n - 2)a_{n}(x - x_{0})^{n - 3} \\
  P'''(x_{0}) &= 2 \cdot 3 \cdot a_{3} 
    \implies a_{3} = \frac{P'''(x_{0})}{3!} \\
  &\cdots \\
  a_{j} &= \frac{P^{(j)}(x_{0})}{j!}
\end{align*}

\begin{equation}
  f(x) = f(x_{0}) + a_{1}(x - x_{0}) + a_{2}(x - x_{0}) +
    a_{3}(x - x_{0})^{2} + \cdots + a_{n} (x - x_{0})^{n} +
    r_{n}(x), \text{ kur }
  \label{teiloras}
\end{equation}
\begin{equation*}
  a_{j} = \frac{f^{(j)}(x_{0})}{j!} 
  \text{ ir }
  r_{n}(x) \text{ – paklaida (arba liekamasis narys).}
\end{equation*}

\begin{prop}
  (Teiloro formulė) Tarkime $f$ yra diferencijuojama $n + 1$ kartą
  intervale $(a; b)$ ir $x, x_{0} \in (a; b)$. Tada
  $\exists c (c \in (x_{0}; x) \text{ arba } c \in (x; x_{0}))$, kad
  \begin{equation*}
    r_{n}(x) = \frac{f^{(n+1)}(c)}{(n+1)!}(x - x_{0})^{n + 1}
  \end{equation*}

  \ref{teiloras} formulę vadiname Teiloro formule.

  \begin{proof}

    Pasiimkime funkciją $\varphi(t)$ tokią, kad:
    \begin{equation*}
      \varphi(t) = f(x) - f(t) - \frac{f'(t)}{1!}(x - t) - \cdots 
        - \frac{f^{(n)}(t)}{n!}(x - t)^{n}.
    \end{equation*}

    $\varphi$ yra bent vieną kartą diferencijuojama pagal kintamąjį $t$,
    nes pagal sąlygą egzistuoja $f^{(n + 1)}(t)$. Taigi 
    (išvestinę skaičiuojame naudodami formulę
    $(f \cdot g)' = f' \cdot g + f \cdot g'$):
    \begin{align*}
      \varphi'_{t} (t) 
      &= 0 - f'(t) - 
      \left( \frac{f''(t)}{1!}(x - t) + \frac{f'(t)}{1!}(-1) \right) 
      - \cdots - 
      \left( \frac{f^{(n+1)}(t)}{n!}(x - t)^{n} + 
        \frac{f^{(n)}(t)}{n!}(x - t)^{n - 1}(-1) \right) \\
% FIXME Sutvarkyti lygiavimą!
      &= - f'(t) - f''(t)(x - t) + f'(t) - \cdots - 
        \frac{f^{(n + 1)}(t)}{n!}(x - t)^{n} + 
        \frac{f^{(n)}(t)}{n!}(x - t)^{n - 1} \\
      &= - \frac{f^{(n + 1)}(t)}{n!}(x - t)^{n}
    \end{align*}

    Paimkime naują funkciją $g$, kuri yra bent vieną kartą 
    diferencijuojama. $\varphi$ ir $g$ tenkina Koši vidutinių 
    reikšmių teoremą. Iš to seka, kad:
    \begin{equation}
      \exists c (c \in (x; x_{0}) \text{ arba } c \in (x_{0}; x)) :
      \frac{\varphi(x) - \varphi(x_{0})}{g(x) - g(x_{0})} = 
      \frac{\varphi'(c)}{g'(c)}.
      \label{_teiloras_01}
    \end{equation}

    Apskaičiuokime reikiamas $\varphi$ funkcijos reikšmes:
    \begin{align*}
      \varphi(x_{0}) &= r_{n}(x) \\
      \varphi(x) &= 0 \\
      \varphi'(t) &= - \frac{f^{(n + 1)}(t)}{n!}(x - t)^{n}
    \end{align*}

    Pasirinkime $g(t) = (x - t)^{n + 1}$ (Pastaba: Pasirinkus kitokią 
    funkciją, gaunama kitokia liekamojo nario forma.) ir apskaičiuokime
    mums reikiamas $g$ funkcijos reikšmes:
    \begin{align*}
      g(x) &= 0 \\
      g(x_{0}) &= (x - x_{0})^{n + 1} \\
      g'_{t}(t) &= - (n + 1)(x - t)^{n}.
    \end{align*}

    Dabar $\varphi$ ir $g$ reikšmes įrašome į (\ref{_teiloras_01}):
    \begin{align*}
      \frac{0 - r_{n}(x)}{-(x - x_{0})^{n + 1}} 
      &= \frac{- \frac{f^{(n + 1)}(c)}{n!}(x - c)^{n}}
        {-(n + 1)(x - c)^{n}} \\
        r_{n}(x) &= \frac{f^{(n + 1)}(c)}{(n + 1)!}(x - x_{0})^{n + 1}.
    \end{align*}

    Gavome Lagranžo formos liekamojo nario išraišką.

  \end{proof}

\end{prop}

\begin{prop}
  Jei $f$ yra n-kartų diferencijuojama taško $x_{0}$ aplinkoje ir
  $f^{(n)}$ yra tolydi taško $x_{0}$ aplinkoje, tai
  \begin{equation*}
    r_{n} (x) = o((x - x_{0})^{n}), \text{ kai } x \to x_{0}.
  \end{equation*}
\end{prop}

\begin{exmp}
  \begin{equation*}
    f(x) = e^{x}, x_{0} = 0
  \end{equation*}

  \begin{align*}
    f^{(n)}(x) &= e^{x} \\
    f^{(n)}(0) &= 1 \\
    f(x) &= 1 + \frac{1}{1!}x^{1} + \frac{1}{2!}x^{2} + 
      \frac{1}{3!}x^{3} + \cdots + \frac{1}{n!}x^{n} + r_{n}(x).
  \end{align*}

  Iš Teiloro formulės:
  \begin{equation*}
    \exists c (c \in (x_{0}; x)) : r_{n}(x) = 
      \frac{e^{c}}{(n + 1)}x^{n + 1}
  \end{equation*}

  Tarkime $f$ nagrinėjame $|x| \leq 1$. Kokia paklaida?

  \begin{equation*}
    |r_{n}(x)| \leq \frac{e^{1}}{(n + 1)!}1 = \frac{e}{(n + 1)!}
  \end{equation*}
\end{exmp}
