\section{Koši kriterijus}

\begin{note}
  Toliau tekste daroma prielaida, jog $f: A \to \RSET$ ir $a$ yra 
  aibės $A$ ribinis taškas.
\end{note}

\begin{prop}
  (Koši kriterijus) Funkcija $f$ turi baigtinę ribą taške $a$
  (taškas $a$ gali būti ir $\infty$) tada ir tik tada:
  \begin{equation}
    \forall \varepsilon (\varepsilon > 0), \exists U_{a} :
    | f(x') - f(x'') | < \varepsilon,
    \forall x', x'' (x',x'' \in U_{a} \cap A, x' \neq a, x'' \neq a)
    \label{kosi}
  \end{equation}

  \begin{proof}
    \hfill \\
    \begin{description}
      \item[Būtinumas] Tarkime, jog $\exists \lim _{x \to a} f(x)$ ir ji
        yra baigtinė. Reikia įrodyti \ref{kosi} teiginį.

        Pažymėkime:
        \begin{equation*}
          \lim _{x \to a} f(x) = b.
        \end{equation*}

        Tada pagal funkcijos ribos apibrėžimą (\ref{limfed}):
        \begin{equation*}
          \lim _{x \to a} f(x) = b \iff 
          \forall \varepsilon (\varepsilon > 0), \exists U_{a}:
          | f(x) - b | < \varepsilon, 
          \forall x (x \in U_{a} \cap A \setminus \{a\}).
        \end{equation*}

        Tada galime įvertinti:
        \begin{align*}
          | f(x') - f(x'') | &= | f(x') - b + b - f(x'') | \\
          &\leq \underbrace{| f(x') - b |}_{ < \varepsilon } +
          \underbrace{| f(x'') - b |}_{ < \varepsilon } \\
          &< 2 \varepsilon, \forall x',x'' 
          (x',x'' \in U_{a} \cap A \setminus \{a\}).
        \end{align*}

      \item[Pakankamumas] Tarkime, kad \ref{kosi} sąlyga teisinga. Reikia 
        įrodyti, kad $\exists \lim _{x \to a} f(x)$ ir kad ji yra baigtinė.

        Kadangi \ref{kosi} teisinga, tai:
        \begin{equation}
          \forall \varepsilon (\varepsilon > 0), \exists U_{a} :
          | f(x') - f(x'') | < \varepsilon, \forall x',x''
          (x',x'' \in U_{a} \cup A \setminus \{a\})
          \label{_kosi_01}
        \end{equation}

        Imame bet kokią seką $\left\{ x_{n} \right\}, x_{n} \to a$. Pagal
        sekos ribos apibrėžimą:
        \begin{equation}
          \forall U_{a}, \exists N : x_{n} \in U_{a}, \forall n (n > \NSET)
          \label{_kosi_02}
        \end{equation}

        Tada iš \ref{_kosi_01} ir \ref{_kosi_02} gauname:
        \begin{equation}
          | f(x_{n}) - f(x_{m}) | < \varepsilon : \forall n, m (n,m > N)
          \label{_kosi_03}
        \end{equation}

        Iš Koši kriterijaus skaičių sekoms ir \ref{_kosi_03} gauname, kad
        seka $\left\{ f(x_{n}) \right\}$ turi baigtinę ribą.

        Tam, kad pilnai būtų patenkintas \ref{limfs} funkcijos ribos 
        apibrėžimas, reikia, kad ir 
        $\left\{ x'_{n} \right\} x'_{n} \to a : \left\{ f(x'_{n}) \right\}$
        turėtų tą pačią ribą, kaip ir $\left\{ f(x_{n}) \right\}$, 
        kai $x_{n} \to a$.

        Sukonstruokime seką
        \begin{equation*}
          x_{1},x'_{1},x_{2},x'_{2},x_{3},x'_{3},\ldots
        \end{equation*}
        ir pažymėkime ją $\left\{ y_{n} \right\}$.

        Kadangi sekos $\left\{ y_{n} \right\}$ posekiai 
        $\left\{ x_{n} \right\}$ ir $\left\{ x'_{n} \right\}$, kurie pilnai 
        padengia visus jos narius, artėja į $a$, tai ir seka 
        $\left\{ y_{n} \right\}$ artėja į $a$.

        Iš Koši kriterijaus skaičių sekoms gauname, jog 
        $\left\{ f(y_{n}) \right\}$ turi baigtinę ribą. Iš skaičių sekos
        ribos apibrėžimo neprieštaringumo gauname, jog 
        $\left\{ f(y_{2n}) \right\} = \left\{ f(x'_{n}) \right\}$ ir
        $\left\{ f(y_{2n+1}) \right\} = \left\{ f(x_{n}) \right\}$ turi tą
        pačią ribą.

    \end{description}
  \end{proof}
\end{prop}
